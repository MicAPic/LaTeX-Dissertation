\chapter{Web 1.0}\label{ch:ch1}

\section{Концептуальное назначение}\label{sec:ch1/sec1}

\[
--
\]

\section{Обобщенная структура}\label{sec:ch1/sec2}

\subsection{Анализ топологии крупных сегментов Веба с помощью модели Бродера “Bow-tie”}\label{subsec:ch1/sec2/sub1}

\textbf{Введение.} В настоящее время все больше организаций, стремясь отразить свою деятельность в Вебе, создают отдельные веб-ресурсы (сайты, группы сайтов) и заботятся об улучшении их рейтингов в информационно-поисковых системах с целью накопления символического капитала, повышения прямых продаж. Важную роль в формировании рейтинга сайтов и локализированных веб-сегментов в целом играют поисковые машины.

С 2006 года основные поисковые машины (например, в мире -- Google, в России -- Яндекс) в своих методах ранжирования наряду с классическими факторами \cite{Kleinberg, BrinPage, Chakrabarti} (частота встречаемости слов, объем цитирования и авторитетность сайтов, частота обновления сайтов и др.) по-новому используют факторы, связанные с поведением пользователя на веб-ресурсах \cite{GuhaKunduBhadra,AntoniouPlegasTsakalidis,FeuerSavevAslam}. Так, действия пользователя, ранее приводившие к повышению рейтингов сайтов (например, частота захода пользователей), ныне могут приводить к “пессимизации” показателей, поскольку поисковые системы становятся более “социальными” и учитывают уже не только профиль пользователя и частоту захода на ресурсы, но также мотивацию посетителей и стратегии их поведения (частоту повторного захода, время нахождения на странице и сайте, логику и маршруты переходов, пользовательские интересы, тип потребляемого контента и многие иные факторы). При ранжировании современные поисковые машины в совокупности учитывают около 300-500 таких факторов.

С другой стороны, существует небольшое количество структурных характеристик Веб-сегментов \cite{ChoRoy}, которые существенно влияют практически на все эти факторы. Имеется значительное число исследований \cite{Kleinberg,ChoRoy,BroderKumarMaghoul,AguilloGranadinoOrtega,StuartThelwallHarries,Chakrabarti,Thelwall,Pechnikov,PechnikovNwohiri}, в которых строится модель фрагментов веба, описывающаяся сравнительно небольшим набором параметров.

Общим направлением исследований авторов является изучение связей между рейтингом сайта и его глобальными характеристиками. По-видимому, эти связи различны для сайтов разных категорий, например, естественнонаучной направленности и гуманитарных. Для выявления существенных различий между категориями сайтов, необходимо исследовать большое количество сайтов с последующей статистической обработкой материала. Авторы располагают разработанной ими аналитической системой для вебометрических исследований \cite{BlekanovSergeevMaksimov,BlekanovSergeevMartynenko}, способной выполнить эту работу. Ввиду чрезвычайно больших затрат машинных ресурсов на подобное исследование, представляет интерес машинный эксперимент, заключающийся в определении затрат на полное исследование типичных сайтов, описание которого и является темой данной статьи.

\textbf{Основная часть.} В эксперименте ставилась задача построения топологии на основе модели Бродера “Bow-tie” \cite{BroderKumarMaghoul,Thelwall} для типичных крупных гуманитарно-ориентированных и естественнонаучно-ориентированных сегментов Веба с последующим вычислением их структурных характеристик и определением произведенных затрат ресурсов.

В качестве гуманитарно-ориентированного веб-сегмента был выбран сайт Факультета журналистики (JF) Санкт-Петербургского государственного университета (СПбГУ) -- www.jf.spbu.ru, естественнонаучно-ориентированного -- сайт факультета Прикладной математики -- процессов управления (APMATH) СПбГУ -- www.apmath.spbu.ru.

С помощью специализированной сборки аналитической системы для вебометрических исследований на основе ядра поискового робота, успешно апробированной в исследованиях \cite{BlekanovSergeev,MaksimovBlekanov} , были выявлены гиперссылочные структуры обоих сайтов, которые описываются в виде двух ориентированных веб-графов \(G_{apmath}\) и \(G_{jf}\), в которых вершины соответствуют страницам, а дуги – соединяющим эти страницы гиперссылкам. Для получения глобальных характеристик использовалась модель Бродера “Bow-tie” \cite{BroderKumarMaghoul} , которая во множестве вершин исследуемого графа выделяет четыре основных подмножества: 1) центральное ядро (компонента сильной связности SCC); 2) истоки (IN); 3) стоки (OUT); 4) “TENDRILS” и “TUBES”. Поиск сильно-связной компоненты выполнялся с помощью алгоритма Косарайю \cite{Sedgewick} (Kosaraju's algorithm), время выполнения которого для разреженных графов (а такими являются исследуемые веб-графы) пропорционально сумме всех их вершин и направленных ребер.

В эксперименте вычислялись следующие характеристики веб-сайтов:
\begin{itemize}
	\item размер центрального ядра \(S_{scc}\);
	\item размер стоков \(S_{out}\) и истоков \(S_{in}\);
	\item размер “TENDRILS” и “TUBES” \(S_{tubes}\);
	\item отношение \(|{S_{scc}| + |S_{in}|}\) к \(|S_{scc}| + |S_{out}|\).
\end{itemize}

Кроме того, для каждой страницы центрального ядра был введен показатель ее связности с другими страницами из ядра:
\[
	\text{\textit{VC}}(u, v) = 
	\begin{cases} 
		1, & \text{\textit{if }} u \rightarrow v,\\
		0, & \text{\textit{else}}
	\end{cases},
	\forall u, v \in S_{scc}
\]

Этот показатель лег в основу усредненной меры связности каждой страницы ядра \(\text{\textit{MA\text{\textit{VC}}}}(u)\) и усредненной меры связности центрального ядра \(\text{\textit{MAC}}(S_{scc})\):
\[
	\text{\textit{MAVC}}(u) = \frac{1}{\lvert S_{scc} \rvert - 1} \sum_{\forall v \in S_{scc}, \; v \neq u}\text{\textit{VC}}(u, v),
\] 
\[
	\text{\textit{MAC}}(S_{scc}) = \frac{1}{|S_{scc}|} \sum_{\forall u \in S_{scc}}\text{\textit{MAVC}}(u).
\]

Приведенные меры были введены для оценки качества связанности каждой топологии, полученной в ходе поставленного эксперимента.

Для оценки трудоемкости эксперимента по каждому исследуемому сайту использовались следующие характеристики:
\begin{itemize}
	\item время (\(T_{total}^{js}\) и \(T_{total}^{apmath}\)), необходимое для выявления топологии и вычисления указанных обобщенных характеристик;
	\item объем памяти (\(\text{\textit{VMem}}_{js}\) и \(\text{\textit{VMem}}_{apmath}\)), требуемый для хранения обработанных данных;
	\item объем Веб трафика \(\text{\textit{Traffic}}_{js}\)  и \(\text{\textit{Traffic}}_{apmath}\)).
\end{itemize}

\textbf{Результаты эксперимента.} Эксперимент проводился на тестовой персональной электронно-вычислительной машине (ПЭВМ) с процессором \textit{Intel Core i7 CPU 950 @ 3.07 GHz x 8}, \textit{12 GB} объемом жесткого диска и операционной системой \textit{Ubuntu 13.10}. Загрузка данных с Интернета выполнялась посредством канала с пропускной способностью до \textit{50 Мбит/с}.

В ходе эксперимента было выявлено, что Веб-граф \(G_{apmath}\) содержит \textit{26 148} вершин и \textit{2 025 909} направленных ребер, а Веб-граф \(G_{jf}\) -- \textit{27 361} вершин и \textit{4 252 424} направленных ребра. На  Рис.~\cref{fig:webGraphs} приведены топологии, построенные для каждого полученного графа с помощью модели Бродера “Bow-tie”.

\begin{figure}[ht]
    \centerfloat{
	        \hfill
	        \subcaptionbox[List-of-Figures entry]{\label{fig:webGraphs-1}}{%
		            \includegraphics[width=0.5\linewidth]{webGraph(apmath)}}
	        \subcaptionbox{\label{fig:webGraphs-2}} страниц от общего их числа), \textit{8.8\%} занимают истоки (IN), - \textit{25.5\%} стоки (OUT), \textit{3.96\%} -- “TENDRILS” и “TUBES” (Рис.~\cref{fig:webGraphs-1}). В то время как доли компонент топологии Веб-графа (Рис.~\cref{fig:webGraphs}\subcaptionref*{fig:webGraphs-2}) приблизительно одинаковые (более \textit{46\%} от общего числа всех страниц) за исключением компоненты OUT (\textit{0.76\%}).

В Таблице \cref{tab:webGraphTable} приведены полученные численные значения структурных характеристик, которые были введены в эксперименте для анализа исследуемых сайтов.

\begin{table} [htbp]%
    \centering
    \caption{Структурные характеристики Веб-графов \(G_{apmath}\) и \(G_{jf}\).}%
    \label{tab:webGraphTable}% label всегда желательно идти после caption
    \renewcommand{\arraystretch}{1.5}%% Увеличение расстояния между рядами, для улучшения восприятия.
    \begin{SingleSpace}
    	\begin{tabulary}{\textwidth}{@{}>{\zz}L >{\zz}C >{\zz}C@{}} %Вертикальные полосы не используются принципиально, как и лишние горизонтальные (допускается по ГОСТ 2.105 пункт 4.4.5) % @{} позволяет прижиматься к краям
		            \toprule     %%% верхняя линейка
		            Характеристика & Топология сайта APMATH &Топология сайта JS \\
		            \midrule %%% тонкий разделитель. Отделяет названия столбцов. Обязателен по ГОСТ 2.105 пункт 4.4.5
		            Мощность множества ядра \(\lvert S_{scc} \rvert\)  & 17142 & 13893    \\
		            Мощность множества IN \(\lvert S_{in} \rvert\)       & 2302     & 13165    \\
		            Мощность множества OUT \(\lvert S_{out} \rvert\)       & 6669     & 206    \\
					Мощность множества \newline TENDRILS и TUBES  \(\lvert S_{tubes} \rvert\)      & 1035    & 12809     \\
					Отношение \(\frac{\lvert S_{scc} \rvert + \lvert S_{in} \rvert}{\lvert S_{scc} \rvert + \lvert S_{out} \rvert}\) & 0.82 & 1.92 \\
					Усредненная мера связности центрального ядра \(\text{\textit{MAC}}(S_{scc})\) & 0.00244 & 0.00193 \\
					Усредненная мера связности всего Веб-графа \textit{MAC(G)} & 0.00106 & 0.00098 \\
		            \bottomrule %%% нижняя линейка
		        \end{tabulary}%
	    \end{SingleSpace}
\end{table}

Затраты ресурсов, требуемых для полного анализа исследуемых в эксперименте сегментов Веба, приведены в Таблице \cref{tab:webGraphCostTable}.

\begin{table}[ht]%
	\caption{Характеристики, оценивающие трудоемкость поставленного эксперимента.}%
	\label{tab:webGraphCostTable}% label всегда желательно идти после caption
    \renewcommand{\arraystretch}{1.6}%% Увеличение расстояния между рядами, для улучшения восприятия.
    \def\tabularxcolumn#1{m{#1}}
    \begin{tabularx}{\textwidth}{@{}>{\raggedright}X>{\centering}m{3.5cm}  >{\centering\arraybackslash}m{3.5cm}@{}}% Вертикальные полосы не используются принципиально, как и лишние горизонтальные (допускается по ГОСТ 2.105 пункт 4.4.5) % @{} позволяет прижиматься к краям
			\toprule     %%% верхняя линейка
			Характеристика & Сайт APMATH & Сайт JS \\
			\midrule %%% тонкий разделитель. Отделяет названия столбцов. Обязателен по ГОСТ 2.105 пункт 4.4.5
			Время, необходимое для построения топологии и вычисления структурных характеристик, \textit{min.}  & \(T_{total}^{apmath} = 672\) &  \(T_{total}^{js} = 271\)  \\
			Среднее время загрузки одной страницы, \textit{min.} & 0.026 & 0.001 \\
			Объем памяти на жестком диске для хранения данных, \textit{MB} & \(\text{\textit{VMem}}_{apmath} = 154.743\) & \(\text{\textit{VMem}}_{js} = 318.463\) \\
			Минимальный объем \newline оперативной памяти \newline для обработки данных, \textit{GB} & \(\approx 1.5\) & \(\approx 1.5\) \\
			Средний объем памяти, требуемый  для полной обработки \newline одной веб-страницы, \textit{kB} & 6.06 & 11.92 \\
			Объем Веб трафика, \textit{GB} &  \(\text{\textit{Traffic}}_{apmath} = 9.66\) &  \(\text{\textit{Traffic}}_{js} = 3.1\) \\
			\bottomrule %%% нижняя линейка
	    \end{tabularx}%
\end{table}

\textbf{Выводы}. Результаты эксперимента подтвердили существенные различия между гиперссылочной структурой представителя сайтов естественнонаучной направленности и структурой представителя гуманитарно-ориентированных сайтов. Так, например, в топологии первого сайта (Рис.~\cref{fig:webGraphs-1}) большая доля всех веб-страниц сосредоточена в центральном ядре, которое имеет большую связность между своими элементами по сравнению с элементами ядра второго сайта (Рис.~\cref{fig:webGraphs}\subcaptionref*{fig:webGraphs-2}). Кроме, того значительное различие компоненты OUT первой топологии в сравнении со второй объясняется наличием в ней большого числа полнотекстовых документов (в формате PDF, DOC, DOCX и др.). Однако, количество гиперссылок в Веб-графе \(G_{jf}\) и характеристик размеров компонент SCC, IN, TUBES/TENDRILS (Таб.~\cref{tab:webGraphTable}) говорят о хорошей коммуникабельности между компонентами топологии Веб-графа \(G_{jf}\).

Для получения статистически достоверных результатов необходимо провести экспериментальные исследования топологий большого числа сайтов для каждой выделенной категории. Опираясь на полученные значения характеристик (Таб.~\cref{tab:webGraphCostTable}), оценивающие трудоемкость поставленного в данной статье эксперимента, можно приблизительно оценить ресурсные затраты на полное исследование любого количества сайтов.

\section{Сбор данных}\label{sec:ch1/sec3}

\subsection{Построение тематико-ориентированных веб-краулеров с использованием обобщенного ядра}\label{subsec:ch1/sec3/sub1}

За последние несколько лет в области информационного Веб-поиска все чаще проводятся исследования \cite{Pechnikov,PechnikovLugovayaChuiko,PechnikovChirkovChuiko}, связанные с развивающимся научным направлением вебометрика (webometrics) \cite{PechnikovLugovayaChuiko}. В частности, методами вебометрики исследуются такие вопросы, как оценка присутствия информационных веб-ресурсов в Вебе, ранжирование веб-сайтов университетов и научных учреждений \cite{Pechnikov,PechnikovChirkovChuiko} или повышение вебометрического рейтинга, введенного испанской группой Cybermetrics Lab (http://www.webometrics.info), различных университетов мира.

К актуальным направлениям вебометрики относятся исследования по анализу и выявлению гиперссылочных структур различных сегментов Веб-пространства (например, академический сегмент Веба, корпоративный и др.). Для получения больших объемов информации о гиперссылках используются Веб-краулеры. Общей задачей таких инструментов является специализированный обход Веба с целью сбора информации или определения гиперссылочной структуры и полезности каких-либо информационных ресурсов \cite{BlekanovBondarenko1,BlekanovBondarenko2}. Имеющиеся в настоящее время Веб-краулеры можно разделить на две группы "--- универсальные и специализированные (тематико-ориентированные). Универсальные обычно избыточны, сильно нагружают сетевые ресурсы и проигрывают специализированным по таким показателям, как время обхода Веб-пространства, производительность обработки информации, а также возможность направленного поиска информационных источников в рамках определенной метрики значимости. В этом смысле преимущества специализированных систем очевидны, однако, их минусом являются значительные затраты на изготовление -- для каждой специализации нужен свой краулер.

Вместе с тем, значительное число функций всех краулеров совпадает \cite{ArasuChoGM,Bar-Ilan,ChoGM,NajorkHeydon}.

В данной статье предлагается модель специализированных Веб-краулеров, состоящих из обобщенного ядра и специализированных дополнений. При таком подходе затраты на изготовление специализированного Веб-краулера минимизируются. Предлагается архитектура обобщенного ядра поискового робота и ее реализация. Ставится эксперимент, в котором сравнивается качество работы Веб-краулеров на основе обобщенного ядра с качеством работы зарубежных аналогов, на примере Heritrix, OpenWebSpider и Methanol Web Crawler.

\textbf{Архитектура обобщенного ядра.} Масштабность Веб-пространства порождают ряд проблем, влияющих на эффективность работы Веб-краулеров любых типов. Анализ известных публикаций \cite{ArasuChoGM,ChoGM,NajorkHeydon,RaghavanGM}, в которых затрагиваются проблемы эффективной работы поисковых роботов, показывает, что любой специализированный поисковый робот должен решать следующие основные проблемы:
\begin{itemize}
	\item \textit{построение специализированного алгоритма обхода Веб-пространства;}
	\item \textit{обновление информационных Веб-ресурсов в общей коллекции документов, собранной Веб-краулером;}
	\item \textit{учет разновидностей форматов Веб-ресурсов;}
	\item \textit{организация информационного поиска по скрытому Вебу} \cite{RaghavanGM}\textit{;}
	\item \textit{минимизация нагрузки на информационный источник;}
	\item \textit{распараллеливание и масштабирование процесса сбора информационных Веб-ресурсов;}
	\item \textit{минимизация нагрузки на каналы связи в Веб-пространстве.}
\end{itemize}

Для их решения была разработана и реализована архитектура обобщенного ядра (Рис.~\cref{fig:kernelArchitecture}), которая может применяться для создания любых типов Веб-краулеров. В основе этой архитектуры лежит краулер-процесс. По структуре \textit{краулер-процесс} является классом, содержащим текущую информацию о состоянии поискового робота (текущий список проанализированных гиперссылок, содержимое Веб-станиц, кэш адресов Веб-страниц, список значимых источников информации), а также набор методов для осуществления поиска и обновления информации в индексе. Данные методы позволяют запускать и останавливать процесс поиска.

Под процессом поиска в рамках рассматриваемой архитектуры понимается итеративный процесс, останавливающийся при выполнении заданных условий (например, число пройденных итераций, количество полученных гиперссылок, сходимость результатов), которые зависят от настроек конфигурации. Каждая итерация данного процесса заключается в последовательном выполнении следующих основных модулей (Рис.~\cref{fig:kernelArchitecture}).

\textit{Многопоточный загрузчик Веб-страниц.} Данный модуль исполняет роль загрузчика Веб-страниц с сервера, на котором они расположены в Веб-пространстве. Также он является менеджером безопасности, который контролирует количество потоков, выделяемое для загрузки всех информационных источников в рамках одной итерации, и не обрабатывает Веб-ресурсы, время отклика которых превышает заданное. Все параметры менеджера загрузок настраиваемые, по умолчанию число потоков равно \textit{10}, а время отклика -- \textit{2 секунды}.

\textit{Модуль извлечения гиперссылок (URLs extraction)}, который ищет дочерние элементы из заданного набора Веб-страниц. Поиск таких элементов осуществляется путем извлечения всех гиперссылок, находящихся в начальном множестве веб-страниц, и их добавления в очередь гиперссылок. Процесс обработки элементов очереди выполняется с помощью набора параллельных и синхронизированных между собой потоков, которые независимо извлекают ссылки и добавляют результаты обработки в общую коллекцию полученных дочерних Веб-страниц.

\textit{Модуль синтаксического анализа Веб-страниц (HTML parser)} -- извлекает текст из загруженных Веб-страниц и определяет его кодировку.

\textit{Нормализация гиперссылок (URLs normalization).} После того, как из начального множества Веб-страниц извлечены все гиперссылки, необходимо отфильтровать их адреса. За данную операцию в обобщенном ядре отвечает модуль фильтрации адресов для Веб-страниц. Основная задача такого модуля заключается в приведении адреса каждого Веб-ресурса к стандартизированному виду по заданным критериям \cite{LeeKimHong} (например, имя хоста и протокола каждого ресурса должны состоять из символов нижнего регистра или адреса всех страниц должны приводиться к одинаковой канонической форме, то есть, заканчиваться символом «/»). Другими словами, модуль фильтрации отсеивает Веб-страницы с некорректными и дублированными адресами, тем самым улучшая производительности работы поискового робота.

\textit{Модуль кэширования.} Операция извлечения содержимого Веб-страниц затрачивает ресурсы системы (интенсивность затрат ресурсов системы зависит от роста числа Веб-страниц), что значительным образом влияет на ее производительность. Поэтому была разработана система кэширования, предназначенная для ускорения процесса извлечения гиперссылок с Веб-страниц за счет повторного использования ранее извлеченного контента. Хранение контента продолжается до тех пор, пока не превышен допустимый порог памяти компьютера, используемой приложением. Данный порог приложение определяет динамически, исходя из имеющегося объема и требований по использованию ресурсов компьютера, а также производительности. При превышении порога система освобождает память путем удаления из кэша контента Веб-страниц, которые реже всего используются.

\textit{Коллекция найденных документов.} Данный модуль хранит информацию обо всех веб-ресурсах и их гиперссылочных структурах, полученных Веб-краулером на каждой итерации краулер-процесса, и предоставляет ее пользователю для дальнейших вебометрических исследований.

\begin{figure}[ht]
	\centerfloat{
		\includegraphics[scale=0.6]{kernelArchitecture}
	}
	\caption{Архитектура обобщенного ядра.}\label{fig:kernelArchitecture}
\end{figure}

Данная архитектура обобщенного ядра предоставляет конфигурируемые настройки интеграции и интерфейсы, использование которых существенно упрощают процесс и минимизируют время добавления в архитектуру новых модулей (Рис.~\cref{fig:kernelModuleLink}), для создания потенциальных внешних модулей (новые алгоритм обхода, синтаксический и семантический анализаторы и др.).

Описанная реализация поискового робота является базовым инструментом для вебометрических исследований, поэтому на ее основе могут создаваться Веб-краулеры со специализированными алгоритмами обхода Веба.

\begin{figure}[ht]
	\centerfloat{
		\includegraphics[scale=0.6]{kernelModuleLink}
	}
	\caption{Связь обобщенного ядра с внешними модулями.}\label{fig:kernelModuleLink}
\end{figure}

\textbf{Архитектурные особенности обобщенного ядра.} Перед экспериментом в данных исследованиях проводилась оценка архитектурных возможностей обобщенного ядра и зарубежных аналогов Веб-краулеров (Heritrix, OpenWebSpider, Methanol Web Crawler) по наличию следующей функциональности (Табл.~\cref{tab:webCrawlerComparison}):
\begin{itemize}
	\item \textit{возможность многопоточной загрузки Веб-страниц;}
	\item \textit{масштабирования процесса сбора Веб-страниц (на программно-аппаратном уровне);}
	\item \textit{минимизация нагрузки на информационные Веб-ресурсы;}
	\item \textit{минимизации нагрузки на каналы связи;}
	\item \textit{гибкость архитектуры (возможность добавлять новые модули, алгоритмы обхода);}
	\item \textit{возможность специализированного обновления информационных Веб-ресурсов в индексе;}
	\item \textit{учет разновидностей форматов Веб-ресурсов;}
	\item \textit{обработка текста Веб-страниц;}
	\item \textit{приспособленность к российскому сегменту Веба;}
	\item \textit{приспособленность к англоязычному сегменту Веба.}
\end{itemize}

\begin{table} [htbp]%
	\centering
	\caption{Сравнение особенностей архитектуры Веб-краулеров.}%
	\label{tab:webCrawlerComparison}% label всегда желательно идти после caption
	\renewcommand{\arraystretch}{1.5}%% Увеличение расстояния между рядами, для улучшения восприятия.
	\begin{SingleSpace}
		\begin{tabulary}{\textwidth}{@{}>{\zz}L >{\zz}C >{\zz}C >{\zz}C >{\zz}C@{}}%Вертикальные полосы не используются принципиально, как и лишние горизонтальные (допускается по ГОСТ 2.105 пункт 4.4.5) % @{} позволяет прижиматься к краям
			\toprule     %%% верхняя линейка
			Функциональность &Heri-\linebreak trix & OpenWeb Spider & Methanol Web Crawler & Обобщенное ядро \\
			\midrule %%% тонкий разделитель. Отделяет названия столбцов. Обязателен по ГОСТ 2.105 пункт 4.4.5
			Многопоточная загрузка &+ &+ &+ &+ \\				
			Масштабирования процесса сбора Веб-ресурсов & -- & -- & -- &+ \\
			Минимизация нагрузки на Веб-ресурсы & + &+ &+ &+ \\
			Минимизации нагрузки на каналы связи &+ & -- & -- &+\\
			Гибкость архитектуры & --& -- & -- &+ \\
			Обновление индекса & -- & -- & -- &+ \\
			Учет разновидностей форматов Веб-ресурсов &+ &+ &+ &+ \\
			Обработка текста Веб-страниц &  -- & -- & -- &+ \\
			Адаптированность к рос. Вебу & -- & -- & -- &+ \\
			Адаптированность к англ. Вебу &+ &+ &+ &+ \\
			\bottomrule %%% нижняя линейка
		\end{tabulary}%
	\end{SingleSpace}
\end{table}

Оценка архитектурных возможностей показала, что обобщенное ядро и Heritrix имеют гибкие настройки (например, объем скачиваемой информации с одного источника, время доступа к серверу Веб-ресурса, количество итераций краулер-процесса и др.), которые при получении и обработке информационных Веб-ресурсов легко позволяют контролировать нагрузку на ресурсы и каналы связи (Табл.~\cref{tab:webCrawlerComparison}). Веб-краулеры OpenWebSpider и Methanol Web Crawler плохо настраиваются (OpenWebSpider конфигурируется в течение \textit{34 минут}, а Methanol Web Crawler -- \textit{30 минут} (Табл.~\cref{tab:crawlersEnglish},~\cref{tab:crawlersRussian})) и при обработке информации источников максимально нагружают каналы связи и используют ресурсы обрабатываемого источника. Вместе с этим зарубежные аналоги плохо адаптированы к российскому Вебу, что подтверждается далее в эксперименте.

\textbf{Эксперимент.} В эксперименте ставилась задача сравнения эффективности работы четырех реализаций Веб-краулеров:
\begin{itemize}
	\item \textit{Веб-краулер на основе обобщенного ядра} с тремя дополнительными модулями: \textit{Module\_BDD}, \textit{Module\_MDD} и \textit{Module\_SDD}, ориентированные на загрузку большого количества Веб-страниц, среднего и малого, соответственно;
	\item \textit{Heritrix (http://crawler.archive.org)} \cite{MohrKimptonStack}, имеющий хорошую производительность при обработке большого количества Веб-страниц;
	\item \textit{Methanol Web Crawler (http://metha-sys.org)}, имеющий хорошую производительность при обработке малого количества с Веб-страниц;
	\item \textit{Open WebSpider (http://www.openwebspider.org)}, имеющий хорошую производительность при обработке небольшого количества с Веб-страниц.
\end{itemize}

Сравнивалась производительность работы и время конфигурирования Веб-краулеров на основе обобщенного ядра с зарубежными аналогами на примере загрузки \textit{2000}, \textit{300000} и \textit{500000} Веб-страниц.

Производительность поисковых роботов проверялась в российском и англоязычном сегментах Веба при одинаковом тестовом наборе данных (ЭВМ, ОС, начальное множество ссылок в очереди и др.). В качестве начальных параметров были заданы начальное множество Веб-страниц для каждого из сегмента (по \textit{10} Веб-страниц), и частота запуска, равная \textit{10}.

\textbf{Результаты эксперимента.} В ходе сравнительного анализа производительности Веб-краулера на основе обобщенного ядра (с подключенными дополнительными модулями \textit{Module\_BDD}, \textit{Module\_MDD} и \textit{Module\_SDD}) с зарубежными аналогами (Heritrix, OpenWebSpider, Methanol Web Crawler) были получены следующие результаты (Табл.~\cref{tab:crawlersEnglish},~\cref{tab:crawlersRussian}):

\begin{table} [htbp]%
	\centering
	\caption{Средняя производительность обработки информации Веб-краулерами в англоязычном сегменте Веб-пространства и время их настройки.}%
	\label{tab:crawlersEnglish}% label всегда желательно идти после caption
	\renewcommand{\arraystretch}{1.5}%% Увеличение расстояния между рядами, для улучшения восприятия.
	\begin{SingleSpace}
		\begin{tabulary}{\textwidth}{@{}>{\zz}L >{\zz}C >{\zz}C >{\zz}C >{\zz}C@{}}%Вертикальные полосы не используются принципиально, как и лишние горизонтальные (допускается по ГОСТ 2.105 пункт 4.4.5) % @{} позволяет прижиматься к краям
			\toprule     %%% верхняя линейка
			Веб-краулеры & Время загрузки 2000 страниц (мин.) & Время загрузки 30000 страниц (мин.) & Время загрузки 50000 страниц (мин.) & Время настройки Веб-краулера (мин.) \\
			\midrule %%% тонкий разделитель. Отделяет названия столбцов. Обязателен по ГОСТ 2.105 пункт 4.4.5
			Heritrix & \textit{1.9} & \textit{8.0} & \textit{11.2} & \textit{20.0} \\				
			OpenWebSpider & \textit{3.0} & \textit{8.4} & \textit{13.0} & \textit{34.0} \\
			Methanol & \textit{1.9} & \textit{8.6} & \textit{14.1} & \textit{30.0} \\			
			Обобщенное ядро с \textit{Module\_BDD} модулем & -- & -- & \textbf{\textit{9.1}} & \textit{4.0}\\
			Обобщенное ядро с \textit{Module\_MDD} модулем & -- & \textbf{\textit{6.0}} & -- & \textit{4.0}\\			
			Обобщенное ядро с \textit{Module\_SDD} модулем & \textbf{\textit{1.7}} & -- & -- & \textit{4.0}\\		
			Обобщенное ядро с модулем анализа текста и ссылок & \textbf{\textit{2.0}} & \textbf{\textit{8.5}} & \textbf{\textit{11.5}} & \textit{4.0}\\		
			\bottomrule %%% нижняя линейка
		\end{tabulary}%
	\end{SingleSpace}
\end{table}

\begin{table} [htbp]%
	\centering
	\caption{Средняя производительность обработки информации Веб-краулерами в российском сегменте Веб-пространства и время их настройки.}%
	\label{tab:crawlersRussian}% label всегда желательно идти после caption
	\renewcommand{\arraystretch}{1.5}%% Увеличение расстояния между рядами, для улучшения восприятия.
	\begin{SingleSpace}
		\begin{tabulary}{\textwidth}{@{}>{\zz}L >{\zz}C >{\zz}C >{\zz}C >{\zz}C@{}}%Вертикальные полосы не используются принципиально, как и лишние горизонтальные (допускается по ГОСТ 2.105 пункт 4.4.5) % @{} позволяет прижиматься к краям
			\toprule     %%% верхняя линейка
			Веб-краулеры & Время загрузки 2000 страниц (мин.) & Время загрузки 30000 страниц (мин.) & Время загрузки 50000 страниц (мин.) & Время настройки Веб-краулера (мин.) \\
			\midrule %%% тонкий разделитель. Отделяет названия столбцов. Обязателен по ГОСТ 2.105 пункт 4.4.5
			Heritrix & \textit{5.0} & \textit{12.9} & \textit{22.3} & \textit{20.0} \\				
			OpenWebSpider & \textit{5.5} & \textit{13.0} & \textit{25.0} & \textit{34.0} \\
			Methanol & \textit{4.2} & \textit{13.5} & \textit{26.5} & \textit{30.0} \\			
			Обобщенное ядро с \textit{Module\_BDD} модулем & -- & -- & \textbf{\textit{10.5}} & \textit{4.0}\\
			Обобщенное ядро с \textit{Module\_MDD} модулем & -- & \textbf{\textit{6.4}} & -- & \textit{4.0}\\			
			Обобщенное ядро с \textit{Module\_SDD} модулем & \textbf{\textit{2.0}} & -- & -- & \textit{4.0}\\		
			Обобщенное ядро с модулем анализа текста и ссылок & \textbf{\textit{3.5}} & \textbf{\textit{9.0}} & \textbf{\textit{13.0}} & \textit{4.0}\\		
			\bottomrule %%% нижняя линейка
		\end{tabulary}%
	\end{SingleSpace}
\end{table}

\begin{itemize}
	\item получено небольшое время добавления внешних модулей (алгоритмы обхода Веб-пространства, синтаксические анализаторы, методы обновления индекса и др.) к обобщенному ядру и их конфигурирования (Табл.~\cref{tab:webCrawlerComparison},~\cref{tab:crawlersEnglish}) по сравнению с зарубежными аналогами, архитектура которых не предусматривает возможность масштабирования и добавления в нее других модулей и время настройки в среднем занимает \textit{28 минут}.
	\item Веб-краулер с обобщенным ядром имеет возможность добавления в него специализированных модулей и их адаптации для тематического поиска (Рис.~\cref{fig:crawlerModes}): классический (рекурсивно обходит все встречающиеся гиперссылки); интересо-ориентированный; ориентированный на популярность; ориентированный на интересы и популярность. В исследованиях \cite{BlekanovBondarenko1,BlekanovBondarenko2} проверялась качество поиска и полезность этих модулей, а также их интеграция с обобщенным ядром. В то время как Heritrix, OpenWebSpider и Methanol Web Crawler используют только классический режим обхода Веб-пространства и не обрабатывают текст источников информации в Вебе, но поддерживают возможность поиска музыкальных, видео файлов, файлов PDF.
	
	\begin{figure}[ht]
		\centerfloat{
			\includegraphics[scale=0.6]{crawlerModes}
		}
		\caption{Режимы поиска Веб-краулера с обобщенным ядром.}\label{fig:crawlerModes}
	\end{figure}
	
	\item Веб-краулеры на основе обобщенного ядра с разными модулями загрузки Веб-страниц (\textit{Module\_BDD}, \textit{Module\_MDD} и \textit{Module\_SDD}) в классическом режиме (обрабатывая информацию только о ссылках) показали высокую производительность среди зарубежных аналогов (Табл.~\cref{tab:crawlersEnglish},~\cref{tab:crawlersRussian}), как в российском, так и зарубежном сегменте Веба. Вместе с этим поисковый робот на основе обобщенного ядра с модулем анализа текста не уступает по производительности зарубежным аналогам, которые не обрабатывают текстовую информацию Веб-страниц, в англоязычном сегменте Веба, а в российском Вебе зарубежные реализации Веб-краулеров сильно уступают в производительности. 
	\item зарубежные аналоги показали низкую производительность обработки информационных Веб-ресурсов в российском сегменте Веба (Табл.~\cref{tab:crawlersEnglish}), а \textit{70\%} ресурсов из этого сегмента Веб-краулеры Heritrix, OpenWebSpider и Methanol Web Crawler не смогли загрузить и обработать. Данный факт объясняется низкой приспособленностью зарубежных аналогов к российскому Вебу.
	\item Среди зарубежных аналогов неплохую эффективность загрузки и обработки \textit{2000}, \textit{300000} и \textit{500000} Веб-страниц показал Веб-краулер Heritrix. Его можно использовать как неплохой инструмент для поиска (с классическим алгоритмом обхода Веба) разнородной информации и добавления ее в индекс.
\end{itemize}

Таким образом, из поставленного эксперимента следует, что Веб-краулер с обобщенным ядром эффективней собирает и обрабатывает информацию с Веб-ресурсов, чем зарубежные аналоги, которые сильно уступают по производительности, гибкости и масштабируемости архитектуры, а также приспособленностью к обработке информации в российском сегменте Веба.

В дальнейшем созданный Веб-краулер с обобщенным ядром планируется использовать в прикладных исследованиях по повышению вебометрических рейтингов университетов, научных институтов и других научных организаций.

\subsection{Веб-краулер как инструмент для вебометрических исследований на примере анализа Веб-пространства СПбГУ}\label{subsec:ch1/sec3/sub2}

\textbf{Введение.} За последние несколько лет в области информационного Веб-поиска все чаще появляются задачи, связанные с развивающимся научным направлением вебометрика (webometrics) \cite{HolmbergThelwall,Pechnikov,PechnikovChirkovChuiko,BlekanovSergeevPechnikov}. К актуальным направлениям вебометрических исследований относятся задачи анализа и выявления гиперссылочных структур различных сегментов Веб-пространства (например, академический сегмент Веба, университетский, и др.), решение которых влияет на качество присутствия этих сегментов в Вебе, на результаты ранжирования поисковых машин (Google, Yandex и др.) или, в случае университетского Веба, на вебометрический рейтинг (http://www.webometrics.info) различных университетов мира \cite{BlekanovSergeevPechnikov}.

 Для получения и обработки больших объемов информации о веб-сайтах и их гиперссылках используются Веб-краулеры (поисковые роботы), общей задачей которых является специализированный обход Веба с целью сбора информации или определения гиперссылочной структуры и полезности каких-либо информационных ресурсов.
 
\textbf{ Эксперимент.} В эксперименте ставилась задача анализа и выявления гиперссылочной структуры Веб-пространства Санкт-Петербургского государственного университета (СПбГУ). 
 
 Для эксперимента использовался программный комплекс обобщенного ядра поискового робота, который обладает высокой гибкостью и масштабируемостью в сравнении с зарубежными аналогами, сильно уступающими в производительности собора и обработки веб-ресурсов и имеющими слабую приспособленность к анализу российского сегмента Веба \cite{BlekanovSergeevMartynenko}. 
 
 К Веб-краулеру с обобщенным ядром дополнительно был разработан и добавлен специализированный алгоритм обхода веб-страниц, который собирает и обрабатывает только страницы из Веб-пространства СПбГУ. В свою очередь пространство СПбГУ состоит из веб-сайта главного домена и сайтов всех его поддоменов (Рис.~\cref{fig:spbuWebSpace}).
 
\begin{figure}[ht]
    \centerfloat{
	        \includegraphics[scale=0.5]{spbuWebSpace}
	    }
    \caption{Веб-пространство СПбГУ.}\label{fig:spbuWebSpace}
\end{figure}

Используя программный комплекс на основе обобщенного ядра поискового робота со специализированным алгоритмом, запущенного с начального множества веб-страниц, требовалось в автоматизированном режиме получить значения следующих показателей, характеризующих гиперссылочную структуру Веб-пространства СПбГУ:
\begin{itemize}
	\item объем Веб-пространства СПбГУ (количество всех различных веб-страниц из Веб-пространства СПбГУ);
	\item количество всех поддоменов из Веб-пространства СПбГУ; 
	\item количество тупиковых (не имеющих ссылок) веб-страниц; 
	\item количество неработающих гиперссылок; 
	\item количество гиперссылок на внешние веб-ресурсы; 
	\item количество поддоменов, связанных с «Главной страницей» главного домена; 
	\item количество поддоменов, несвязанных с «Главной страницей» главного домена; 
	\item количество страниц, имеющие гиперссылки на «Главную страницу» главного домена; 
	\item количество страниц, не имеющие гиперссылки на «Главную страницу» главного домена; 
	\item гиперссылочная структура Веб-пространства СПбГУ в виде матрицы смежности. 
\end{itemize}

В качестве начального множества веб-страниц, с которого Веб-краулер запускал процесс сбора и обработки веб-ресурсов, брался URL-адрес главного веб-сайта СПбГУ -- «http://www.spbu.ru/».

\textbf{Результаты эксперимента.} В ходе эксперимента всего Веб-краулером было обработано и проанализировано \textit{6\space429\space963} гиперссылки, которые содержались на страницах Веб-пространства СПбГУ. Из них: объем ссылок на внешние источники информации равен \textit{507\space168}, а объем внутренних ссылок (на страницы главного домена сайта СПбГУ и его поддоменов) -- \textit{5\space922\space795}. Кроме того, были получены следующие результаты (Табл.~\cref{tab:indicators}):

\begin{table} [htbp]%
	\centering
	\caption{}%
	\label{tab:indicators}% label всегда желательно идти после caption
	\renewcommand{\arraystretch}{1.5}%% Увеличение расстояния между рядами, для улучшения восприятия.
	\begin{SingleSpace}
		\begin{tabulary}{\textwidth}{@{}>{\zz}L >{\zz}C@{}} %Вертикальные полосы не используются принципиально, как и лишние горизонтальные (допускается по ГОСТ 2.105 пункт 4.4.5) % @{} позволяет прижиматься к краям
			\toprule     %%% верхняя линейка
			Показатель & Значение показателя  \\
			\midrule %%% тонкий разделитель. Отделяет названия столбцов. Обязателен по ГОСТ 2.105 пункт 4.4.5
			объем Веб-пространства СПбГУ & 71688 \\				
			количество всех поддоменов из Веб-пространства СПбГУ & 315 \\
			количество поддоменов, сильно-связанных с «Главной страницей» главного домена & 280 \\
			количество поддоменов, несвязанных с «Главной страницей» главного домена & 35 \\
			количество тупиковых веб-страниц & 12516 \\
			количество неработающих гиперссылок & 680 \\
			количество страниц, имеющие гиперссылки на «Главную страницу» главного домена & 10632 \\
			количество страниц, не имеющие гиперссылки на «Главную страницу» главного домена & 61056 \\
			\bottomrule %%% нижняя линейка
		\end{tabulary}%
	\end{SingleSpace}
\end{table}

Также стоит отметить, что для простоты исследований и получения численных значений показателей было предложено хранить выявленную гиперссылочную структуру Веб-пространства СПбГУ в виде матрицы смежности A размерностью [71688\(\times\)71688]. С помощью нее и ее транспонированной формы удобно определять связанность между страницами и в дальнейшем оптимизировать структуру ссылок. 

\textbf{Общие выводы.} По результатам эксперимента видно, что \textit{17,5\%} веб-страниц из всех страниц Веб-пространства СПбГУ занимают тупиковые страницы, которые, например, не учитывает поисковая машина Google в своем алгоритме ранжирования PageRank, что негативно влияет на коммуникабельность всего Веб-пространство университета в целом. 

Также из эксперимента наблюдается плохая связанность страниц Веб-пространства СПбГУ и его поддоменов со страницами главного сайта университета: только \textit{14,8\%} всех веб-страниц имеет ссылки на страницы главного сайта, а \textit{35} доменов со всеми своими страницами и вовсе никак не связаны с ним. 

Стоит отметить и большое количество неработающих гиперссылок (\textit{680} уникальных гиперссылок), которые неоднократно повторяются по всему Веб-пространству СПбГУ, тем самым снижают доверие пользователей к ресурсам сайтов университета. 

Таким образом, проведенный эксперимент демонстрирует слабую связанность и коммуникабельность внутренних ресурсов Веб-пространства СПбГУ, что влечет и его слабую позицию в вебометрическом рейтинге сайтов университетов мира.

\section{Анализ гиперссылочной структуры ресурсов}\label{sec:markup}

\subsection{Теоретико-графовые характеристики в вебометрических исследованиях внутренней топологии крупных сегментов Веба}\label{subsec:ch1/sec4/sub1}

\textbf{Введение.} В настоящее время стремительно развивается молодое научное направление вебометрика \cite{AlmindIngwersen,Thelwall,Pechnikov}, которая занимается исследованиями количественных и качественных характеристик топологии (гиперссылочной структуры) различных веб-сегментов. Основоположниками таких исследований являются испанская группа Cybermetrics Lab, которая разработала вебометрический рейтинг сайтов различных крупных организаций \cite{RankingWeb} (таких как научно-образовательные учреждения, больницы, научно-исследовательские центры и т. п.).  

Одной из актуальных задач вебометрики является задача анализа внутренней топологии различных сайтов \cite{Thelwall,BlekanovSergeevMaksimov,MaksimovBlekanov,BlekanovSergeevMaksimovBOWTIE}, а также выявление критериев оценки качества веб-сегментов и сравнение этих сегментов по полученным оценкам.  

Для решения вышеуказанной задачи в статье авторами рассмотрены известные теоретико-графовые характеристики веб-графов и разработан комплекс программ на их основе. Данный программный комплекс используется для вычисления указанных выше характеристик, их сравнения на разных веб-сегментах большой размерности, а также для визуализации результатов сравнительного анализа.  

\textbf{Эксперимент.} В работе ставился эксперимент, в котором разработанный авторами комплекс программ для выявления значимых страниц внутренней топологии крупных веб-сегментов (с помощью используемых теоретико-графовых характеристик) и их визуализации апробировался на крупных сайтах университетского Веба. 

Получение внутренней топологии веб-ресурсов в эксперименте выполнялось с помощью программно-аналитического комплекса для вебометрических исследований, основанного на обобщенном ядре поискового робота \cite{BlekanovSergeevMartynenko} и успешно апробированного в исследованиях \cite{BlekanovSergeevMaksimov,MaksimovBlekanov,BlekanovSergeevMaksimovBOWTIE}.

В качестве теоретико-графовых характеристик сайтов были выбраны несколько мер центральности и связности веб-графов, а именно:
\begin{enumerate}
	\item Степень вершины (Degree Centrality) -- показатель, указывающий для каждой страницы количество страниц, связанных с ней. В веб-графе вычисляются полустепень захода (indegree, количество входящих ссылок) и полустепень исхода (outdegree, количество исходящих ссылок) ссылок \cite{WassermanFaust,OrtegaAguillo}.
		\item Мера центральности Betweenneess -- показатель, указывающий насколько часто данная веб-страница лежит на кратчайшем пути между всеми парами страниц сайта \cite{WassermanFaust,OrtegaAguillo}.
		\item Мера центральности Closeness -- среднее расстояние от данной веб-страницы до всех остальных страниц сайта \cite{WassermanFaust}.
		\item Мера центральности PageRank -- показатель важности страницы. Чем выше ее показатель, тем она важнее \cite{PageBrinMotwani}.
		\item Мера связности p-Cliques -- подграф, который представляет собой полный граф. p -- количество вершин в данном подграфе \cite{OrtegaAguillo}.
		\item Мера связности k-Cores -- подграф, в котором каждая страница связана по крайне мере с k другими страницами в этом подграфе \cite{OrtegaAguillo}.
\end{enumerate}

А также использовались такие общие показатели топологии, как:
\begin{enumerate}
	\item Расстояние (Distance) -- средняя длина всех кратчайших путей в графе \cite{MaksimovBlekanov}. 
	\item Диаметр (Diameter) -- длина самого большого кратчайшего пути в графе \cite{MaksimovBlekanov}.
\end{enumerate}

В качестве сайтов университетского Веба были взяты следующие сайты из Мирового вебометрического рейтинга университетов \cite{RankingWeb}:
\begin{enumerate}
	\item Сайт, занимающий первое место в общем рейтинге:
		\begin{enumerate}
				\item Сайт Гарвардского университета -- ГУ (www.harvard.edu). 
			\end{enumerate}
	\item Сайты, занимающие первые места в рейтинге по Российской Федерации: 
		\begin{enumerate}
				\item Сайт Московского государственного университета -- МГУ (www.msu.ru). 
				\item Сайт Санкт-Петербургского государственного университета -- СПбГУ (www.spbu.ru).
			\end{enumerate}
\end{enumerate}

При получении внутренней топологии выше указанных сайтов учитывались только главные их домены (поддомены не рассматривались). 

Используя вышеуказанный программный комплекс, требовалось получить визуальное представление топологии сайтов университетов в виде композиции наилучших значимых веб-страниц по всем мерам центральности (по каждой мере выбирался топ-10 страниц с наилучшими весами), а также построить и оценить расстояние от главной страницы до этих страниц.

\textbf{Результаты эксперимента.} В ходе эксперимента для выбранных сайтов университетского Веба были получены следующие значения мер связности и общих показателей топологии (Табл.~\cref{tab:uniPagesInnerTopology}):

\begin{table} [htbp]%
	\centering
	\caption{Теоретико-графовые характеристики внутренней топологии университетских сайтов.}%
	\label{tab:uniPagesInnerTopology}% label всегда желательно идти после caption
	\renewcommand{\arraystretch}{1.5}%% Увеличение расстояния между рядами, для улучшения восприятия.
	\begin{SingleSpace}
		\begin{tabulary}{\textwidth}{@{}>{\zz}L >{\zz}C >{\zz}C >{\zz}C@{}} %Вертикальные полосы не используются принципиально, как и лишние горизонтальные (допускается по ГОСТ 2.105 пункт 4.4.5) % @{} позволяет прижиматься к краям
			\toprule     %%% верхняя линейка
			Теоретико-графовые характеристики & ГУ & МГУ & СПбГУ\\
			\midrule %%% тонкий разделитель. Отделяет названия столбцов. Обязателен по ГОСТ 2.105 пункт 4.4.5
			Количество страниц & \textit{451} & \textit{45 557} & \textit{51 945} \\				
			Количество ссылок, принадлежащих главному домену & \textit{27 615} & \textit{2 043 221} & \textit{4 862 750} \\
			Общее количество ссылок & \textit{72 355} & \textit{2 143 474} & \textit{5 017 815} \\
			Расстояние & \textit{3.04} & \textit{7.19} & \textit{6.02} \\
			Расстояние & \textit{7} & \textit{30} & \textit{18} \\
			Среднее значение indegree, outdegree & \textit{13.59} & \textit{28.1} & \textit{69.79} \\
			Количество вершин в p-Cliques & \textit{26} & \textit{10} & \textit{144} \\
			\bottomrule %%% нижняя линейка
		\end{tabulary}%
	\end{SingleSpace}
\end{table}

Композиция всех наилучших (по заданным мерам центральности) значимых веб-страниц исследуемых сайтов представлена на рис.~\cref{fig:harvardUComposition}-\cref{fig:spbUComposition}. На рисунках~\cref{fig:harvardUComposition}-\cref{fig:spbUComposition} введены следующие обозначения цветов: 
\begin{itemize}
	\item Красный -- главная страница сайта; 
	\item  Зеленый -- полустепень исхода; 
	\item Бирюзовый -- полустепень захода; 
	\item Синий -- мера Betweenness; 
	\item Розовый -- мера Closeness; 
	\item Оранжевый -- мера PageRank.
\end{itemize}

\begin{figure}[ht]
	\centerfloat{
		\includegraphics[scale=0.5]{harvardUComposition}
	}
	\caption{Композиция всех наилучших значимых веб-страниц сайта ГУ.}\label{fig:harvardUComposition}
\end{figure}

\begin{figure}[ht]
	\centerfloat{
		\includegraphics[scale=0.5]{moscowSUComposition}
	}
	\caption{Композиция всех наилучших значимых веб-страниц сайта МГУ.}\label{fig:moscowSUComposition}
\end{figure}

\begin{figure}[ht]
	\centerfloat{
		\includegraphics[scale=0.5]{spbUComposition}
	}
	\caption{Композиция всех наилучших значимых веб-страниц сайта СПбГУ.}\label{fig:spbUComposition}
\end{figure}

\textbf{Общие выводы.} По результатам эксперимента можно сделать следующие выводы:
\begin{enumerate}
	\item Сайт ГУ представляет собой навигационный сайт, перенеся большую часть информации на свои поддомены (ссылки главного домена составляют \textit{38.2\%} от общего числа ссылок). Данное явление подтверждается тем, что веб-страницы по полустепени исхода у ГУ лежат близко к главной странице (глубина 2-4), в то время как у МГУ и СПбГУ они лежат на определенном удалении (8-11 у МГУ и 6-9 у СПбГУ).
	\item Показатели СПбГУ (такие как расстояние, диаметр, количество вершин в p-Cliques, положение значимых вершин относительно начальной страницы) лучше, чем таковые у МГУ. Это значит, что сайт СПбГУ показывает более высокую связность, нежели сайт МГУ.
	\item Авторитетные страницы расположены на следующем уровне глубины после главной страницы, о чем говорит их высокий вес PageRank.
	\item Значимые веб-страницы по мере Betweenness у всех сайтов лежат в радиусе пяти ссылок от главной страницы, однако у сайта МГУ они расположены последовательно. Это объясняется тем, что множество кратчайших путей проходит через эти страницы. При отказе одной из таких страниц увеличиваются размеры множества кратчайших путей, а также появляется риск полной потери связей с множеством страниц.
	\item Значимые веб-страницы по мере Closeness у ГУ лежат на уровнях 1 и 3, у МГУ -- на уровне 4 и 5, у СПбГУ -- на уровне 3. Это показывает, что рассмотренные сайты централизованы недалеко от главной страницы.
\end{enumerate}

Также эксперимент показал некоторую особенность главных страниц рассмотренных сайтов, а именно:
\begin{itemize}
	\item главная страница ГУ не попала ни в один из топ-10;
	\item главная страница МГУ попала лишь в топ-10 по степени захода;
	\item главная страница СПбГУ попала сразу в три топ-10: по степени захода, по мере Betweenness и по мере PageRank. 
\end{itemize}

В дальнейшем планируется расширить эксперимент, проверив эргономические параметры значимых веб-страниц, выделенных программным комплексом, у заданных сайтов. 

\FloatBarrier
