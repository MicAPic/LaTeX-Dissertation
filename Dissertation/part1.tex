\chapter{Web 1.0}\label{ch:ch1}

\section{Концептуальное назначение}\label{sec:ch1/sec1}

\[
--
\]

\section{Обобщенная структура}\label{sec:ch1/sec2}

\subsection{Анализ топологии крупных сегментов Веба с помощью модели Бродера “Bow-tie”}\label{subsec:ch1/sec2/sub1}

\paragraph{Введение.} В настоящее время все больше организаций, стремясь отразить свою деятельность в Вебе, создают отдельные веб-ресурсы (сайты, группы сайтов) и заботятся об улучшении их рейтингов в информационно-поисковых системах с целью накопления символического капитала, повышения прямых продаж. Важную роль в формировании рейтинга сайтов и локализированных веб-сегментов в целом играют поисковые машины.

С 2006 года основные поисковые машины (например, в мире -- Google, в России -- Яндекс) в своих методах ранжирования наряду с классическими факторами \cite{Kleinberg, BrinPage, Chakrabarti} (частота встречаемости слов, объем цитирования и авторитетность сайтов, частота обновления сайтов и др.) по-новому используют факторы, связанные с поведением пользователя на веб-ресурсах \cite{GuhaKunduBhadra,AntoniouPlegasTsakalidis,FeuerSavevAslam}. Так, действия пользователя, ранее приводившие к повышению рейтингов сайтов (например, частота захода пользователей), ныне могут приводить к “пессимизации” показателей, поскольку поисковые системы становятся более “социальными” и учитывают уже не только профиль пользователя и частоту захода на ресурсы, но также мотивацию посетителей и стратегии их поведения (частоту повторного захода, время нахождения на странице и сайте, логику и маршруты переходов, пользовательские интересы, тип потребляемого контента и многие иные факторы). При ранжировании современные поисковые машины в совокупности учитывают около 300-500 таких факторов.

С другой стороны, существует небольшое количество структурных характеристик Веб-сегментов \cite{ChoRoy}, которые существенно влияют практически на все эти факторы. Имеется значительное число исследований \cite{Kleinberg,ChoRoy,BroderKumarMaghoul,AguilloGranadinoOrtega,StuartThelwallHarries,Chakrabarti,Thelwall,Pechnikov,PechnikovNwohiri}, в которых строится модель фрагментов веба, описывающаяся сравнительно небольшим набором параметров.

Общим направлением исследований авторов является изучение связей между рейтингом сайта и его глобальными характеристиками. По-видимому, эти связи различны для сайтов разных категорий, например, естественнонаучной направленности и гуманитарных. Для выявления существенных различий между категориями сайтов, необходимо исследовать большое количество сайтов с последующей статистической обработкой материала. Авторы располагают разработанной ими аналитической системой для вебометрических исследований \cite{BlekanovSergeevMaksimov,BlekanovSergeevMartynenko}, способной выполнить эту работу. Ввиду чрезвычайно больших затрат машинных ресурсов на подобное исследование, представляет интерес машинный эксперимент, заключающийся в определении затрат на полное исследование типичных сайтов, описание которого и является темой данной статьи.

\paragraph{Основная часть.} В эксперименте ставилась задача построения топологии на основе модели Бродера “Bow-tie” \cite{BroderKumarMaghoul,Thelwall} для типичных крупных гуманитарно-ориентированных и естественнонаучно-ориентированных сегментов Веба с последующим вычислением их структурных характеристик и определением произведенных затрат ресурсов.

В качестве гуманитарно-ориентированного веб-сегмента был выбран сайт Факультета журналистики (JF) Санкт-Петербургского государственного университета (СПбГУ) -- www.jf.spbu.ru, естественнонаучно-ориентированного -- сайт факультета Прикладной математики -- процессов управления (APMATH) СПбГУ -- www.apmath.spbu.ru.

С помощью специализированной сборки аналитической системы для вебометрических исследований на основе ядра поискового робота, успешно апробированной в исследованиях \cite{BlekanovSergeev,MaksimovBlekanov} , были выявлены гиперссылочные структуры обоих сайтов, которые описываются в виде двух ориентированных веб-графов \(G_{apmath}\) и \(G_{jf}\), в которых вершины соответствуют страницам, а дуги – соединяющим эти страницы гиперссылкам. Для получения глобальных характеристик использовалась модель Бродера “Bow-tie” \cite{BroderKumarMaghoul}, которая во множестве вершин исследуемого графа выделяет четыре основных подмножества: 1) центральное ядро (компонента сильной связности SCC); 2) истоки (IN); 3) стоки (OUT); 4) “TENDRILS” и “TUBES”. Поиск сильно-связной компоненты выполнялся с помощью алгоритма Косарайю \cite{Sedgewick} (Kosaraju's algorithm), время выполнения которого для разреженных графов (а такими являются исследуемые веб-графы) пропорционально сумме всех их вершин и направленных ребер.

В эксперименте вычислялись следующие характеристики веб-сайтов:
\begin{itemize}
	\item размер центрального ядра \(S_{scc}\);
	\item размер стоков \(S_{out}\) и истоков \(S_{in}\);
	\item размер “TENDRILS” и “TUBES” \(S_{tubes}\);
	\item отношение \({\lvert S_{scc} \rvert + \lvert S_{in} \rvert}\) к \(\lvert S_{scc} \rvert + \lvert S_{out} \rvert\).
\end{itemize}

Кроме того, для каждой страницы центрального ядра был введен показатель ее связности с другими страницами из ядра:
\[
	\text{\textit{VC}}(u, v) = 
	\begin{cases} 
		1, & \text{\textit{if }} u \rightarrow v,\\
		0, & \text{\textit{else}}
	\end{cases},
	\forall u, v \in S_{scc}
\]

Этот показатель лег в основу усредненной меры связности каждой страницы ядра \(\text{\textit{MA\text{\textit{VC}}}}(u)\) и усредненной меры связности центрального ядра \(\text{\textit{MAC}}(S_{scc})\):
\[
	\text{\textit{MAVC}}(u) = \frac{1}{\lvert S_{scc} \rvert - 1} \sum_{\forall v \in S_{scc}, \; v \neq u}\text{\textit{VC}}(u, v),
\] 
\[
	\text{\textit{MAC}}(S_{scc}) = \frac{1}{|S_{scc}|} \sum_{\forall u \in S_{scc}}\text{\textit{MAVC}}(u).
\]

Приведенные меры были введены для оценки качества связанности каждой топологии, полученной в ходе поставленного эксперимента.

Для оценки трудоемкости эксперимента по каждому исследуемому сайту использовались следующие характеристики:
\begin{itemize}
	\item время (\(T_{total}^{js}\) и \(T_{total}^{apmath}\)), необходимое для выявления топологии и вычисления указанных обобщенных характеристик;
	\item объем памяти (\(\text{\textit{VMem}}_{js}\) и \(\text{\textit{VMem}}_{apmath}\)), требуемый для хранения обработанных данных;
	\item объем Веб трафика \(\text{\textit{Traffic}}_{js}\) и \(\text{\textit{Traffic}}_{apmath}\)).
\end{itemize}

\paragraph{Результаты эксперимента.} Эксперимент проводился на тестовой персональной электронно-вычислительной машине (ПЭВМ) с процессором \textit{Intel Core i7 CPU 950 @ 3.07 GHz x 8}, \textit{12 GB} объемом жесткого диска и операционной системой \textit{Ubuntu 13.10}. Загрузка данных с Интернета выполнялась посредством канала с пропускной способностью до \textit{50 Мбит/с}.

В ходе эксперимента было выявлено, что Веб-граф \(G_{apmath}\) содержит \textit{26 148} вершин и \textit{2 025 909} направленных ребер, а Веб-граф \(G_{jf}\) -- \textit{27 361} вершин и \textit{4 252 424} направленных ребра. На рис.~\cref{fig:webGraphs} приведены топологии, построенные для каждого полученного графа с помощью модели Бродера “Bow-tie”.

\begin{figure}[ht]
    \centerfloat{
	        \hfill
	        \subcaptionbox[List-of-Figures entry]{\label{fig:webGraphs-1}}{%
		            \includegraphics[width=0.5\linewidth]{webGraph(apmath)}}
	        \subcaptionbox{\label{fig:webGraphs-2}} страниц от общего их числа), \textit{8.8\%} занимают истоки (IN), - \textit{25.5\%} стоки (OUT), \textit{3.96\%} -- “TENDRILS” и “TUBES” (рис.~\cref{fig:webGraphs-1}). В то время как доли компонент топологии Веб-графа (рис.~\cref{fig:webGraphs}\subcaptionref*{fig:webGraphs-2}) приблизительно одинаковые (более \textit{46\%} от общего числа всех страниц) за исключением компоненты OUT (\textit{0.76\%}).

В табл.~\cref{tab:webGraphTable} приведены полученные численные значения структурных характеристик, которые были введены в эксперименте для анализа исследуемых сайтов.

\begin{table} [htbp]%
    \centering
    \caption{Структурные характеристики Веб-графов \(G_{apmath}\) и \(G_{jf}\).}%
    \label{tab:webGraphTable}% label всегда желательно идти после caption
    \renewcommand{\arraystretch}{1.5}%% Увеличение расстояния между рядами, для улучшения восприятия.
    \begin{SingleSpace}
    	\begin{tabulary}{\textwidth}{@{}>{\zz}L >{\zz}C >{\zz}C@{}} %Вертикальные полосы не используются принципиально, как и лишние горизонтальные (допускается по ГОСТ 2.105 пункт 4.4.5) % @{} позволяет прижиматься к краям
		            \toprule     %%% верхняя линейка
		            Характеристика & Топология сайта APMATH &Топология сайта JS \\
		            \midrule %%% тонкий разделитель. Отделяет названия столбцов. Обязателен по ГОСТ 2.105 пункт 4.4.5
		            Мощность множества ядра \(\lvert S_{scc} \rvert\)  & 17142 & 13893    \\
		            Мощность множества IN \(\lvert S_{in} \rvert\)       & 2302     & 13165    \\
		            Мощность множества OUT \(\lvert S_{out} \rvert\)       & 6669     & 206    \\
					Мощность множества \newline TENDRILS и TUBES  \(\lvert S_{tubes} \rvert\)      & 1035    & 12809     \\
					Отношение \(\frac{\lvert S_{scc} \rvert + \lvert S_{in} \rvert}{\lvert S_{scc} \rvert + \lvert S_{out} \rvert}\) & 0.82 & 1.92 \\
					Усредненная мера связности центрального ядра \(\text{\textit{MAC}}(S_{scc})\) & 0.00244 & 0.00193 \\
					Усредненная мера связности всего Веб-графа \textit{MAC(G)} & 0.00106 & 0.00098 \\
		            \bottomrule %%% нижняя линейка
		        \end{tabulary}%
	    \end{SingleSpace}
\end{table}

Затраты ресурсов, требуемых для полного анализа исследуемых в эксперименте сегментов Веба, приведены в табл.~\cref{tab:webGraphCostTable}.

\begin{table}[ht]%
	\caption{Характеристики, оценивающие трудоемкость поставленного эксперимента.}%
	\label{tab:webGraphCostTable}% label всегда желательно идти после caption
    \renewcommand{\arraystretch}{1.6}%% Увеличение расстояния между рядами, для улучшения восприятия.
    \def\tabularxcolumn#1{m{#1}}
    \begin{tabularx}{\textwidth}{@{}>{\raggedright}X>{\centering}m{3.5cm}  >{\centering\arraybackslash}m{3.5cm}@{}}% Вертикальные полосы не используются принципиально, как и лишние горизонтальные (допускается по ГОСТ 2.105 пункт 4.4.5) % @{} позволяет прижиматься к краям
			\toprule     %%% верхняя линейка
			Характеристика & Сайт APMATH & Сайт JS \\
			\midrule %%% тонкий разделитель. Отделяет названия столбцов. Обязателен по ГОСТ 2.105 пункт 4.4.5
			Время, необходимое для построения топологии и вычисления структурных характеристик, \textit{min.}  & \(T_{total}^{apmath} = 672\) &  \(T_{total}^{js} = 271\)  \\
			Среднее время загрузки одной страницы, \textit{min.} & 0.026 & 0.001 \\
			Объем памяти на жестком диске для хранения данных, \textit{MB} & \(\text{\textit{VMem}}_{apmath} = 154.743\) & \(\text{\textit{VMem}}_{js} = 318.463\) \\
			Минимальный объем \newline оперативной памяти \newline для обработки данных, \textit{GB} & \(\approx 1.5\) & \(\approx 1.5\) \\
			Средний объем памяти, требуемый  для полной обработки \newline одной веб-страницы, \textit{kB} & 6.06 & 11.92 \\
			Объем Веб трафика, \textit{GB} &  \(\text{\textit{Traffic}}_{apmath} = 9.66\) &  \(\text{\textit{Traffic}}_{js} = 3.1\) \\
			\bottomrule %%% нижняя линейка
	    \end{tabularx}%
\end{table}

\paragraph{Выводы.} Результаты эксперимента подтвердили существенные различия между гиперссылочной структурой представителя сайтов естественнонаучной направленности и структурой представителя гуманитарно-ориентированных сайтов. Так, например, в топологии первого сайта (рис.~\cref{fig:webGraphs-1}) большая доля всех веб-страниц сосредоточена в центральном ядре, которое имеет большую связность между своими элементами по сравнению с элементами ядра второго сайта (рис.~\cref{fig:webGraphs}\subcaptionref*{fig:webGraphs-2}). Кроме, того значительное различие компоненты OUT первой топологии в сравнении со второй объясняется наличием в ней большого числа полнотекстовых документов (в формате PDF, DOC, DOCX и др.). Однако, количество гиперссылок в Веб-графе \(G_{jf}\) и характеристик размеров компонент SCC, IN, TUBES/TENDRILS (Таб.~\cref{tab:webGraphTable}) говорят о хорошей коммуникабельности между компонентами топологии Веб-графа \(G_{jf}\).

Для получения статистически достоверных результатов необходимо провести экспериментальные исследования топологий большого числа сайтов для каждой выделенной категории. Опираясь на полученные значения характеристик (Таб.~\cref{tab:webGraphCostTable}), оценивающие трудоемкость поставленного в данной статье эксперимента, можно приблизительно оценить ресурсные затраты на полное исследование любого количества сайтов.

\subsection{Research of University Sites Internal Links Distribution}\label{subsec:ch1/sec2/sub2}

\paragraph{Intoduction.} The paper is concerned with the university sites structural properties study. An oriented graph is used as a generally accepted website structure representation. Nodes of the graph correspond to webpages, while arcs correspond to links between webpages. Consider that webpage does not have its own graph structure in terms of this graph.

Any site is a part of the World Wide Web, so we have to consider how to lay emphasis on a particular site.

First of all, technically speaking, the site has a domain name and actually only webpages corresponding to the domain can be referred to the organization website. Secondly, we consider subdomains to be external in relation to the website main domain. Thirdly, all links that are relevant to the main site webpages and subdomain webpages or other domain webpages are also considered to be external and are not part of the website.

University websites are usually both extremely large and significantly vary in size. A typical website has hundreds or thousands of pages, whereas there are also sites including 25, 10 thousand and 1 million pages (an approximate number of website pages can be estimated using Google's «site:» search term). Since the websites are so large, the task to determine structural, thematic, ergonomic characteristics of the site becomes truly laborious.

Knowledge about these characteristics will allow owners of large scientific and educational organizations’ websites to analyze the quality of their web resources presence in Web. Particularly, studies of structural characteristics will allow to monitor connections between web resources of organizations, to capture target groups of users and to evaluate site relevancy for them, to organize effective logistics of hyperlink structure \cite{StuartThelwallHarries,Thelwall,PechnikovNwohiri,PechnikovNwohiri2}. Ergonomic characteristics are estimating the adaptation of web resources of organizations to relevancy and ranking evaluation criteria of top search engines \cite{Huang,BodrunovaYakuninSmolin,LangvilleMeyer}. Enhancement of these characteristics will help stimulating user interest for every site and extend the time users spend on the sites which is the most important factor of relevancy in search engines rankings. However, estimating these characteristics for big sites is not an easy task in terms of time, computing power and system resources.

It seems natural to try to find an approximate solution to this issue, looking at a relatively small part of the site. Though, when you try to do this, you will get stuck at the point where you need to know what part of the site has already been looked through. To answer this question, you still need to look through the whole site to find out the total number of pages.

In our paper \cite{SergeevBlekanovMaksimov} we offered a method for website size (the number of webpages and links) estimation by its fraction (10\%). By collecting 11 university websites it has been shown that this method gives an estimate site webpages number with an average error of 40\%. The analysis was based on studies in \cite{BarabasiAlbert,BroderKumarMaghoul}, revealing that the pages are distributed according to the power law by the number of incoming links. We developed the idea that university sites are special in some way and that they have a different exponent. The preliminary analysis revealed that the changes of the exponent of the distribution law in our formulas \cite{SergeevBlekanovMaksimov} could have a significant influence on the site’s webpages and hyperlinks count estimation error. So, we performed an experiment using hyperlink structure of 97 universities from top 500 Webometrics  \cite{RankingWeb} rating. The information has been collected by means of authors' specialized web crawler \cite{BlekanovSergeevMartynenko}, adapted for webometric studies. Search engines like Google, Yahoo, etc. do not provide this kind of information.

\paragraph{Experiment.} \textit{Goal.} The experiment specifies the distribution law of university sites webpages by the number of incoming links. We assume that the distribution law is a power law. We need to find out the exponent for a power law.

\textit{Selecting university websites.} Since the websites of the world's leading universities are in great interests of the Webometrics rating, we decided to take first 500 sites into account. In order to reduce the huge amount of work, we have randomly chosen 100 university sites (from 1 to 500 of Webometrics rating), about 2-3 sites from every dozen.

Using web crawler designed by authors all pages and websites internal links of 97 universities have been collected and processed \cite{WebometricsLab}. We were unable to collect data from three sites due to their special layout. As a result of random selection universities were distributed by region as follows:
\begin{itemize}
	\item North America -- 42
	\item South America -- 2
	\item Europe -- 36
	\item Asia -- 13
	\item Australia and New Zealand -- 4
\end{itemize}

\textit{Experimental results.} Preliminary analysis displayed that university sites can be divided into several distinct groups by the number of webpages (Table~\cref{tab:sitesGroupedByNoOfPages}). This distinction caused some dubiety, whether or not it is possible to calculate the universal distribution exponent. 

\begin{table} [htbp]%
	\centering
	\caption{Sites grouped by the number of pages.}%
	\label{tab:sitesGroupedByNoOfPages}% label всегда желательно идти после caption
	\renewcommand{\arraystretch}{1.5}%% Увеличение расстояния между рядами, для улучшения восприятия.
	\begin{SingleSpace}
		\begin{tabulary}{\textwidth}{@{}>{\zz}L >{\zz}C@{}} %Вертикальные полосы не используются принципиально, как и лишние горизонтальные (допускается по ГОСТ 2.105 пункт 4.4.5) % @{} позволяет прижиматься к краям
			\toprule     %%% верхняя линейка
			Pages count & Sites count  \\
			\midrule %%% тонкий разделитель. Отделяет названия столбцов. Обязателен по ГОСТ 2.105 пункт 4.4.5
			<1 300 & \textit{12} \\
			2 000-9 200 & \textit{10} \\
			10 000-25 000 & \textit{16} \\ 
			25 000-61 000 & \textit{12} \\
			61 000-100 000 &  \textit{7} \\
			100 000-400 000 & \textit{15} \\
			400 000-1 000 000  & \textit{9} \\ 
			>1 000 000  & \textit{16} \\
			\bottomrule %%% нижняя линейка
		\end{tabulary}%
	\end{SingleSpace}
\end{table}

We formed a plane for each university site, where horizontal coordinate is the number of incoming links, and vertical coordinate is the proportion of the page’s number. A set of points was applied to the plane. A power approximating curve was drawn, that helped to determine the exponent. We omitted some of the points while drawing the curve. These were points with the proportion of page count less than \(10^{-5}\). A
typical picture is shown in figure~\cref{fig:uniCollegeLondonCurve}. 

\begin{figure}[ht]
	\centerfloat{
		\includegraphics[scale=0.5]{uniCollegeLondonCurve}
	}
	\caption{Approximation curve for University College London site.}\label{fig:uniCollegeLondonCurve}
\end{figure}

We created a plane for each exponent of all university websites pages’ power distribution law where X axis represents the number of pages and Y axis represents exponent module. (Fig.~\cref{fig:sitesExponent}). 

\begin{figure}[ht]
	\centerfloat{
		\includegraphics[scale=0.4]{sitesExponent}
	}
	\caption{The exponent for different sites (X axis represents the number of pages of these sites). Black dot separates sites from first two groups from the rest of the sites.}\label{fig:sitesExponent}
\end{figure}

Figure 2 shows that sites from first and second groups have a vastly varying exponent that makes them stand out from other sites. This is most likely because their structure is influenced by two factors: management decisions regarding the structure and random spontaneous structure influences by active users. When a website is small (less than 10 thousand pages), management decisions influence is prevailing. And since every university has its own unique administrative unit, sites have different structural characteristics and different incoming link distributions. But when a website is big, the number of random structure influences becomes so large, that the role of management becomes less significant. 

\begin{figure}[ht]
	\centerfloat{
		\includegraphics[scale=0.4]{bigUniAveragedDistribution}
	}
	\caption{Averaged distribution for big university sites. Solid line represents power law distribution from A. Broder and R. Kumar (2000), Barabasi and Albert (1999) papers. Circles show variation areas of the averaged distribution values.}\label{fig:bigUniAveragedDistribution}
\end{figure}

\begin{figure}[ht]
	\centerfloat{
		\includegraphics[scale=0.4]{allUniAveragedDistribution}
	}
	\caption{Averaged distribution for all university websites. Solid line represents power law distribution from A. Broder and R. Kumar (2000), Barabasi and Albert (1999) papers.}\label{fig:allUniAveragedDistribution}
\end{figure}

We calculated two average exponents (Fig.~\cref{fig:bigUniAveragedDistribution},~\cref{fig:allUniAveragedDistribution}): \(k_\textit{uni\_all} = 1.8\) for all 97 sites and \(k_\textit{uni\_max} = 1.9\) just for big sites (75 sites with more than 10 thousand pages). These exponents differ from those obtained in \cite{BarabasiAlbert,BroderKumarMaghoul}. 

Figure~\cref{fig:bigUniAveragedDistribution} shows averaged distribution for all big sites. Two splashes (highlighted with circles) attract our attention: large deviations from the approximating curve. The first -- from 8 to 10 incoming links, the second -- around 69-71. The second can obviously be explained by the fact that the main page for most sites is here. The first is harder to explain. Perhaps, such a number of links is typical for main pages of different structural divisions of universities.

Figure~\cref{fig:allUniAveragedDistribution} shows that despite the fact there are many splashes due to considering small sized sites in terms of page count, the average exponent stays the same -- 1.8.

\paragraph{Conclusions and future work.} Our main results: 

\begin{itemize}
	\item All pages of 97 sites have been collected
	\item All internal incoming links for each page have been
	collected
	\item Exponent of power distribution for each university site
	has been calculated.
	\item The proximity of large (more than 10 thousand pages)
	sites distributions (\(k_\textit{uni\_max} = 1.9\)) and difference in
	distributions of small sites have been revealed (with
	the exponent between 0.0 and 5.0).
	\item The averaged approximating distribution curve (the
	exponent \(k_\textit{uni\_all} = 1.8\)) has been constructed and
	analyzed.
	\item Average exponent values have been calculated.
	Hyperlink structure of university sites was found to be
	different from \cite{BarabasiAlbert,BroderKumarMaghoul} in terms of the exponent.
\end{itemize}

The final results will be used to improve the algorithms for estimation of the size of any university site (page and hyperlink count) by studying just a part of the site. In the future, we plan to estimate structural, thematic and ergonomic characteristics of university websites by studying a randomly selected part. 

\subsection{Индекс для обработки и хранения гиперссылочных структур крупных Веб-сегментов}\label{subsec:ch1/sec2/sub3}

\paragraph{Введение.} В настоящее время все больше организаций заботятся о качестве собственных веб-ресурсов в сети Интернет, а именно: мониторят и увеличивают рейтинги в поисковых системах; увеличивают индекс цитирования в Вебе; увеличивают вебометрический рейтинг \cite{RankingWeb} (в случае научно-общеобразовательных учреждений); улучшают коммуникабельность гиперссылочной структуры веб-ресурсов; улучшают эргономические свойства веб-ресурсов (удобочитаемость шрифтов, удобство глубины экрана, продолжительность пребывания пользователя на сайте, доступности целевого контента, эффективности брендинга в визуальном оформлении сайта и т. п.).

Исследованиями количественных и качественных характеристик гиперссылочной структуры занимается активно-развивающееся научное направление -- вебометрика \cite{Pechnikov,PechnikovNwohiri}. Такие исследования вебресурсов называются вебометрическими. Местом апробации различных вебометрических методов, алгоритмов и программно-аналитических инструментов является университетский Веб \cite{Pechnikov,MaksimovBlekanov,BlekanovSergeevMaksimovBOWTIE}, который включает крупные веб-сегменты, состоящие из сайтов научно-образовательных учреждений.

Важной особенностью сайтов университетского Веба является сложная разветвленная гиперссылочная структура большой размерности. В среднем, подобные веб-структуры ссылок содержат более 50 000 узлов и более 2 000 000 связей между ними \cite{Pechnikov,BlekanovSergeevMaksimov}. В связи с этим одной из актуальных задач вебометрики является задача индексирования гиперссылочных структур больших Веб-сегментов, решение которой в целом позволило бы существенно ускорить различные вебометрические методы и алгоритмы анализа крупных сайтов, а также систематизировать хранение таких структур.

В данной работе в качестве решения поставленной выше задачи разработана индексная структура, основанная на использовании принципов инвертированного индекса.

\paragraph{Теоретическая часть.} Идея разработанной индексной структуры схожа с идеей реализации прямого и обратного индекса для полнотекстового поиска \cite{ManningRaghavanSchutze}. Индекс состоит из следующих частей:

\begin{enumerate}
	\item Словарь, содержащий множество страниц исследуемого веб-сегмента, которым при обработке присваиваются уникальные целочисленные идентификаторы. Ключом в словаре является хэш от URL-адреса, а значением -- идентификатор для этого адреса. Данная структура позволяет осуществлять поиск и добавление идентификатора для заданного URL-адреса страницы в среднем за \(O(1\)). Для эффективного поиска (за время \(O(1)\)) URL-адреса любой страницы по идентификатору вместе со словарем в индексе используется специальный массив, в котором ключом является идентификатор, а значением -- адрес страницы. Эти структуры расходуют \(O(n)\) памяти, где n -- число индексируемых страниц.
	\item Множества входящих и исходящих ссылок для каждой страницы. Множество исходящих ссылок заранее известно, а для нахождения множества входящих ссылок выполняется следующая процедура: для каждой веб-страницы \(T\) пробегаем по всем исходящим ссылкам и для каждой исходящей ссылки \(L\) добавляем для страницы \(L\) входящую ссылку \(T\). Данные множества ссылок хранятся в массивах переменной длины.
\end{enumerate}

Кроме того, в индексе реализована возможность мониторинга гиперссылочной структуры исследуемого сайта во времени, которая основана на хранении в базе данных различных сессий индексирования данного сайта.

Теоретическая оценка времени построения разработанного индекса составляет \(O(n)\).

\paragraph{Эксперимент.} Ставился эксперимент, в котором разработанный индекс для обработки и хранения гиперссылочных структур апробировался на крупных сайтах научно-образовательных учреждений. В качестве сайтов из мирового вебометрического рейтинга вузов \cite{RankingWeb} были выбраны два сайта российских университетов с наивысшим рейтингом:

\begin{enumerate}
	\item Сайт Московского государственного университета.
	\item Сайт Санкт-Петербургского государственного университета.
\end{enumerate}

Для сбора и выявления гиперссылочной структуры веб-ресурсов использовался прототип программно-аналитической системы для вебометрических исследований, основанной на обобщенном ядре поискового робота и успешно апробированной в исследованиях \cite{MaksimovBlekanov,BlekanovSergeevMaksimovBOWTIE}.

Эксперимент условно был разделен на три части. В первой части оценивалось время построения индекса для сайтов МГУ и СПбГУ. Во второй -- время работы алгоритма Косарайю (алгоритм нахождения компонент сильной связности графа), использующего данные из построенного индекса. В третьей -- производилась оценка скорости нахождения изменений в гиперссылочной структуре сайта МГУ (информация об удаленных и добавленных ссылках), собранной программным комплексом в разные моменты времени (в первой сессии сайт был проиндексирован полностью, во второй -- проиндексирована лишь некоторая его часть, чтобы гарантировать наличие изменений в гиперссылочной структуре сайта).

\paragraph{Результаты эксперимента.} Результаты оценки времени построения индексов для сайтов МГУ и СПбГУ приведены в таблице~\cref{tab:kosarajuMSUAndSPbU}.

\begin{table} [htbp]%
	\centering
	\caption{Оценка времени построения индекса и работы алгоритма Косарайю.}%
	\label{tab:kosarajuMSUAndSPbU}% label всегда желательно идти после caption
	\renewcommand{\arraystretch}{1.5}%% Увеличение расстояния между рядами, для улучшения восприятия.
	\begin{SingleSpace}
		\begin{tabulary}{\textwidth}{@{}>{\zz}L >{\zz}C >{\zz}C@{}} %Вертикальные полосы не используются принципиально, как и лишние горизонтальные (допускается по ГОСТ 2.105 пункт 4.4.5) % @{} позволяет прижиматься к краям
			\toprule     %%% верхняя линейка
			Параметры оценки & Сайт МГУ & Сайт СПбГУ  \\
			\midrule %%% тонкий разделитель. Отделяет названия столбцов. Обязателен по ГОСТ 2.105 пункт 4.4.5
			Количество обработанных индексом веб-страниц & 45 379 & 127 682 \\
			Количество обработанных индексом ссылок & 3 330 258 & 16 713 138 \\
			Индексирование исходящих ссылок, сек. & 31,4 & 101,4 \\
			Индексирование входящих ссылок, сек. & 0,71 & 3,7 \\
			Время выполнения алгоритма Косарайю, сек. & 0,051 & 0,12 \\
			\bottomrule %%% нижняя линейка
		\end{tabulary}%
	\end{SingleSpace}
\end{table}

Результаты анализа изменения гиперссылочной структуры сайта
МГУ между двумя сессиями сбора приведены в таблице~\cref{tab:timeMSUAndSPbU}.

\begin{table} [htbp]%
	\centering
	\caption{Оценка времени построения индекса и работы алгоритма Косарайю.}%
	\label{tab:timeMSUAndSPbU}% label всегда желательно идти после caption
	\renewcommand{\arraystretch}{1.5}%% Увеличение расстояния между рядами, для улучшения восприятия.
	\begin{SingleSpace}
		\begin{tabulary}{\textwidth}{@{}>{\zz}L >{\zz}C >{\zz}C@{}} %Вертикальные полосы не используются принципиально, как и лишние горизонтальные (допускается по ГОСТ 2.105 пункт 4.4.5) % @{} позволяет прижиматься к краям
			\toprule     %%% верхняя линейка
			Параметры оценки & Сессия сбора №1 & Сессия сбора №2  \\
			\midrule %%% тонкий разделитель. Отделяет названия столбцов. Обязателен по ГОСТ 2.105 пункт 4.4.5
			Количество веб-страниц & 45 379 & 41 314 \\
			Количество ссылок & 3 330 258 & 3 120 322 \\
			Количество добавленных ссылок & & 641 563\\
			Количество удаленных ссылок & & 851 499 \\
			Поиск добавленных и удаленных ссылок, сек. & & 5,18 \\
			\bottomrule %%% нижняя линейка
		\end{tabulary}%
	\end{SingleSpace}
\end{table}

\paragraph{Заключение.} Таким образом, результаты эксперимента показывают, что разработанный индекс имеет высокую скорость построения. Существенное различие скорости индексирования исходящих и скорости индексирования входящих ссылок (табл.~\cref{tab:kosarajuMSUAndSPbU}) объясняется тем, что при индексировании исходящих ссылок приходится обращаться к базе данных, хранящей предварительно собранную программным комплексом веб-структуру сайта, в то время, как при построении индекса для входящих ссылок используется уже построенный индекс исходящих ссылок. Созданная авторами индексная структура также показала высокую скорость взаимодействия с алгоритмом Косарайю, что в результате обеспечило быстрый поиск компонент сильной связности (табл.~\cref{tab:kosarajuMSUAndSPbU}).
	
Процедура получение информации из индекса по двум сессиям (табл.~\cref{tab:timeMSUAndSPbU}), хранящим состояние гиперссылочной структуры сайта МГУ, для определение разницы между ними, работает несколько медленнее, чем при взаимодействии индекса с алгоритмом Касарайю. Это связанно с загрузкой сессий из базы данных. Однако, это несущественно влияет на анализ гиперссылочных структур крупных сегментов веба.

\section{Сбор данных}\label{sec:ch1/sec3}

\subsection{Построение тематико-ориентированных веб-краулеров с использованием обобщенного ядра}\label{subsec:ch1/sec3/sub1}

За последние несколько лет в области информационного Веб-поиска все чаще проводятся исследования \cite{Pechnikov,PechnikovLugovayaChuiko,PechnikovChirkovChuiko}, связанные с развивающимся научным направлением вебометрика (webometrics) \cite{PechnikovLugovayaChuiko}. В частности, методами вебометрики исследуются такие вопросы, как оценка присутствия информационных веб-ресурсов в Вебе, ранжирование веб-сайтов университетов и научных учреждений \cite{Pechnikov,PechnikovChirkovChuiko} или повышение вебометрического рейтинга, введенного испанской группой Cybermetrics Lab (http://www.webometrics.info), различных университетов мира.

К актуальным направлениям вебометрики относятся исследования по анализу и выявлению гиперссылочных структур различных сегментов Веб-пространства (например, академический сегмент Веба, корпоративный и др.). Для получения больших объемов информации о гиперссылках используются Веб-краулеры. Общей задачей таких инструментов является специализированный обход Веба с целью сбора информации или определения гиперссылочной структуры и полезности каких-либо информационных ресурсов \cite{BlekanovBondarenko1,BlekanovBondarenko2}. Имеющиеся в настоящее время Веб-краулеры можно разделить на две группы "--- универсальные и специализированные (тематико-ориентированные). Универсальные обычно избыточны, сильно нагружают сетевые ресурсы и проигрывают специализированным по таким показателям, как время обхода Веб-пространства, производительность обработки информации, а также возможность направленного поиска информационных источников в рамках определенной метрики значимости. В этом смысле преимущества специализированных систем очевидны, однако, их минусом являются значительные затраты на изготовление -- для каждой специализации нужен свой краулер.

Вместе с тем, значительное число функций всех краулеров совпадает \cite{ArasuChoGM,Bar-Ilan,ChoGM,NajorkHeydon}.

В данной статье предлагается модель специализированных Веб-краулеров, состоящих из обобщенного ядра и специализированных дополнений. При таком подходе затраты на изготовление специализированного Веб-краулера минимизируются. Предлагается архитектура обобщенного ядра поискового робота и ее реализация. Ставится эксперимент, в котором сравнивается качество работы Веб-краулеров на основе обобщенного ядра с качеством работы зарубежных аналогов, на примере Heritrix, OpenWebSpider и Methanol Web Crawler.

\paragraph{Архитектура обобщенного ядра.} Масштабность Веб-пространства порождают ряд проблем, влияющих на эффективность работы Веб-краулеров любых типов. Анализ известных публикаций \cite{ArasuChoGM,ChoGM,NajorkHeydon,RaghavanGM}, в которых затрагиваются проблемы эффективной работы поисковых роботов, показывает, что любой специализированный поисковый робот должен решать следующие основные проблемы:
\begin{itemize}
	\item \textit{построение специализированного алгоритма обхода Веб-пространства;}
	\item \textit{обновление информационных Веб-ресурсов в общей коллекции документов, собранной Веб-краулером;}
	\item \textit{учет разновидностей форматов Веб-ресурсов;}
	\item \textit{организация информационного поиска по скрытому Вебу} \cite{RaghavanGM}\textit{;}
	\item \textit{минимизация нагрузки на информационный источник;}
	\item \textit{распараллеливание и масштабирование процесса сбора информационных Веб-ресурсов;}
	\item \textit{минимизация нагрузки на каналы связи в Веб-пространстве.}
\end{itemize}

Для их решения была разработана и реализована архитектура обобщенного ядра (рис.~\cref{fig:kernelArchitecture}), которая может применяться для создания любых типов Веб-краулеров. В основе этой архитектуры лежит краулер-процесс. По структуре \textit{краулер-процесс} является классом, содержащим текущую информацию о состоянии поискового робота (текущий список проанализированных гиперссылок, содержимое Веб-станиц, кэш адресов Веб-страниц, список значимых источников информации), а также набор методов для осуществления поиска и обновления информации в индексе. Данные методы позволяют запускать и останавливать процесс поиска.

Под процессом поиска в рамках рассматриваемой архитектуры понимается итеративный процесс, останавливающийся при выполнении заданных условий (например, число пройденных итераций, количество полученных гиперссылок, сходимость результатов), которые зависят от настроек конфигурации. Каждая итерация данного процесса заключается в последовательном выполнении следующих основных модулей (рис.~\cref{fig:kernelArchitecture}).

\textit{Многопоточный загрузчик Веб-страниц.} Данный модуль исполняет роль загрузчика Веб-страниц с сервера, на котором они расположены в Веб-пространстве. Также он является менеджером безопасности, который контролирует количество потоков, выделяемое для загрузки всех информационных источников в рамках одной итерации, и не обрабатывает Веб-ресурсы, время отклика которых превышает заданное. Все параметры менеджера загрузок настраиваемые, по умолчанию число потоков равно \textit{10}, а время отклика -- \textit{2 секунды}.

\textit{Модуль извлечения гиперссылок (URLs extraction)}, который ищет дочерние элементы из заданного набора Веб-страниц. Поиск таких элементов осуществляется путем извлечения всех гиперссылок, находящихся в начальном множестве веб-страниц, и их добавления в очередь гиперссылок. Процесс обработки элементов очереди выполняется с помощью набора параллельных и синхронизированных между собой потоков, которые независимо извлекают ссылки и добавляют результаты обработки в общую коллекцию полученных дочерних Веб-страниц.

\textit{Модуль синтаксического анализа Веб-страниц (HTML parser)} -- извлекает текст из загруженных Веб-страниц и определяет его кодировку.

\textit{Нормализация гиперссылок (URLs normalization).} После того, как из начального множества Веб-страниц извлечены все гиперссылки, необходимо отфильтровать их адреса. За данную операцию в обобщенном ядре отвечает модуль фильтрации адресов для Веб-страниц. Основная задача такого модуля заключается в приведении адреса каждого Веб-ресурса к стандартизированному виду по заданным критериям \cite{LeeKimHong} (например, имя хоста и протокола каждого ресурса должны состоять из символов нижнего регистра или адреса всех страниц должны приводиться к одинаковой канонической форме, то есть, заканчиваться символом «/»). Другими словами, модуль фильтрации отсеивает Веб-страницы с некорректными и дублированными адресами, тем самым улучшая производительности работы поискового робота.

\textit{Модуль кэширования.} Операция извлечения содержимого Веб-страниц затрачивает ресурсы системы (интенсивность затрат ресурсов системы зависит от роста числа Веб-страниц), что значительным образом влияет на ее производительность. Поэтому была разработана система кэширования, предназначенная для ускорения процесса извлечения гиперссылок с Веб-страниц за счет повторного использования ранее извлеченного контента. Хранение контента продолжается до тех пор, пока не превышен допустимый порог памяти компьютера, используемой приложением. Данный порог приложение определяет динамически, исходя из имеющегося объема и требований по использованию ресурсов компьютера, а также производительности. При превышении порога система освобождает память путем удаления из кэша контента Веб-страниц, которые реже всего используются.

\textit{Коллекция найденных документов.} Данный модуль хранит информацию обо всех веб-ресурсах и их гиперссылочных структурах, полученных Веб-краулером на каждой итерации краулер-процесса, и предоставляет ее пользователю для дальнейших вебометрических исследований.

\begin{figure}[ht]
	\centerfloat{
		\includegraphics[scale=0.6]{kernelArchitecture}
	}
	\caption{Архитектура обобщенного ядра.}\label{fig:kernelArchitecture}
\end{figure}

Данная архитектура обобщенного ядра предоставляет конфигурируемые настройки интеграции и интерфейсы, использование которых существенно упрощают процесс и минимизируют время добавления в архитектуру новых модулей (рис.~\cref{fig:kernelModuleLink}), для создания потенциальных внешних модулей (новые алгоритм обхода, синтаксический и семантический анализаторы и др.).

Описанная реализация поискового робота является базовым инструментом для вебометрических исследований, поэтому на ее основе могут создаваться Веб-краулеры со специализированными алгоритмами обхода Веба.

\begin{figure}[ht]
	\centerfloat{
		\includegraphics[scale=0.6]{kernelModuleLink}
	}
	\caption{Связь обобщенного ядра с внешними модулями.}\label{fig:kernelModuleLink}
\end{figure}

\paragraph{Архитектурные особенности обобщенного ядра.} Перед экспериментом в данных исследованиях проводилась оценка архитектурных возможностей обобщенного ядра и зарубежных аналогов Веб-краулеров (Heritrix, OpenWebSpider, Methanol Web Crawler) по наличию следующей функциональности (табл.~\cref{tab:webCrawlerComparison}):
\begin{itemize}
	\item \textit{возможность многопоточной загрузки Веб-страниц;}
	\item \textit{масштабирования процесса сбора Веб-страниц (на программно-аппаратном уровне);}
	\item \textit{минимизация нагрузки на информационные Веб-ресурсы;}
	\item \textit{минимизации нагрузки на каналы связи;}
	\item \textit{гибкость архитектуры (возможность добавлять новые модули, алгоритмы обхода);}
	\item \textit{возможность специализированного обновления информационных Веб-ресурсов в индексе;}
	\item \textit{учет разновидностей форматов Веб-ресурсов;}
	\item \textit{обработка текста Веб-страниц;}
	\item \textit{приспособленность к российскому сегменту Веба;}
	\item \textit{приспособленность к англоязычному сегменту Веба.}
\end{itemize}

\begin{table} [htbp]%
	\centering
	\caption{Сравнение особенностей архитектуры Веб-краулеров.}%
	\label{tab:webCrawlerComparison}% label всегда желательно идти после caption
	\renewcommand{\arraystretch}{1.5}%% Увеличение расстояния между рядами, для улучшения восприятия.
	\begin{SingleSpace}
		\begin{tabulary}{\textwidth}{@{}>{\zz}L >{\zz}C >{\zz}C >{\zz}C >{\zz}C@{}}%Вертикальные полосы не используются принципиально, как и лишние горизонтальные (допускается по ГОСТ 2.105 пункт 4.4.5) % @{} позволяет прижиматься к краям
			\toprule     %%% верхняя линейка
			Функциональность &Heri-\linebreak trix & OpenWeb Spider & Methanol Web Crawler & Обобщенное ядро \\
			\midrule %%% тонкий разделитель. Отделяет названия столбцов. Обязателен по ГОСТ 2.105 пункт 4.4.5
			Многопоточная загрузка &+ &+ &+ &+ \\				
			Масштабирования процесса сбора Веб-ресурсов & -- & -- & -- &+ \\
			Минимизация нагрузки на Веб-ресурсы & + &+ &+ &+ \\
			Минимизации нагрузки на каналы связи &+ & -- & -- &+\\
			Гибкость архитектуры & --& -- & -- &+ \\
			Обновление индекса & -- & -- & -- &+ \\
			Учет разновидностей форматов Веб-ресурсов &+ &+ &+ &+ \\
			Обработка текста Веб-страниц &  -- & -- & -- &+ \\
			Адаптированность к рос. Вебу & -- & -- & -- &+ \\
			Адаптированность к англ. Вебу &+ &+ &+ &+ \\
			\bottomrule %%% нижняя линейка
		\end{tabulary}%
	\end{SingleSpace}
\end{table}

Оценка архитектурных возможностей показала, что обобщенное ядро и Heritrix имеют гибкие настройки (например, объем скачиваемой информации с одного источника, время доступа к серверу Веб-ресурса, количество итераций краулер-процесса и др.), которые при получении и обработке информационных Веб-ресурсов легко позволяют контролировать нагрузку на ресурсы и каналы связи (табл.~\cref{tab:webCrawlerComparison}). Веб-краулеры OpenWebSpider и Methanol Web Crawler плохо настраиваются (OpenWebSpider конфигурируется в течение \textit{34 минут}, а Methanol Web Crawler -- \textit{30 минут} (табл.~\cref{tab:crawlersEnglish},~\cref{tab:crawlersRussian})) и при обработке информации источников максимально нагружают каналы связи и используют ресурсы обрабатываемого источника. Вместе с этим зарубежные аналоги плохо адаптированы к российскому Вебу, что подтверждается далее в эксперименте.

\paragraph{Эксперимент.} В эксперименте ставилась задача сравнения эффективности работы четырех реализаций Веб-краулеров:
\begin{itemize}
	\item \textit{Веб-краулер на основе обобщенного ядра} с тремя дополнительными модулями: \textit{Module\_BDD}, \textit{Module\_MDD} и \textit{Module\_SDD}, ориентированные на загрузку большого количества Веб-страниц, среднего и малого, соответственно;
	\item \textit{Heritrix (http://crawler.archive.org)} \cite{MohrKimptonStack}, имеющий хорошую производительность при обработке большого количества Веб-страниц;
	\item \textit{Methanol Web Crawler (http://metha-sys.org)}, имеющий хорошую производительность при обработке малого количества с Веб-страниц;
	\item \textit{Open WebSpider (http://www.openwebspider.org)}, имеющий хорошую производительность при обработке небольшого количества с Веб-страниц.
\end{itemize}

Сравнивалась производительность работы и время конфигурирования Веб-краулеров на основе обобщенного ядра с зарубежными аналогами на примере загрузки \textit{2000}, \textit{300000} и \textit{500000} Веб-страниц.

Производительность поисковых роботов проверялась в российском и англоязычном сегментах Веба при одинаковом тестовом наборе данных (ЭВМ, ОС, начальное множество ссылок в очереди и др.). В качестве начальных параметров были заданы начальное множество Веб-страниц для каждого из сегмента (по \textit{10} Веб-страниц), и частота запуска, равная \textit{10}.

\paragraph{Результаты эксперимента.} В ходе сравнительного анализа производительности Веб-краулера на основе обобщенного ядра (с подключенными дополнительными модулями \textit{Module\_BDD}, \textit{Module\_MDD} и \textit{Module\_SDD}) с зарубежными аналогами (Heritrix, OpenWebSpider, Methanol Web Crawler) были получены следующие результаты (табл.~\cref{tab:crawlersEnglish},~\cref{tab:crawlersRussian}):

\begin{table} [htbp]%
	\centering
	\caption{Средняя производительность обработки информации Веб-краулерами в англоязычном сегменте Веб-пространства и время их настройки.}%
	\label{tab:crawlersEnglish}% label всегда желательно идти после caption
	\renewcommand{\arraystretch}{1.5}%% Увеличение расстояния между рядами, для улучшения восприятия.
	\begin{SingleSpace}
		\begin{tabulary}{\textwidth}{@{}>{\zz}L >{\zz}C >{\zz}C >{\zz}C >{\zz}C@{}}%Вертикальные полосы не используются принципиально, как и лишние горизонтальные (допускается по ГОСТ 2.105 пункт 4.4.5) % @{} позволяет прижиматься к краям
			\toprule     %%% верхняя линейка
			Веб-краулеры & Время загрузки 2000 страниц (мин.) & Время загрузки 30000 страниц (мин.) & Время загрузки 50000 страниц (мин.) & Время настройки Веб-краулера (мин.) \\
			\midrule %%% тонкий разделитель. Отделяет названия столбцов. Обязателен по ГОСТ 2.105 пункт 4.4.5
			Heritrix & 1.9 & 8.0 & 11.2 & 20.0 \\				
			OpenWebSpider & 3.0 & 8.4 & 13.0 & 34.0 \\
			Methanol & 1.9 & 8.6 & 14.1 & 30.0 \\			
			Обобщенное ядро с \textit{Module\_BDD} модулем & -- & -- & \textbf{9.1} & 4.0\\
			Обобщенное ядро с \textit{Module\_MDD} модулем & -- & \textbf{6.0} & -- & 4.0\\			
			Обобщенное ядро с \textit{Module\_SDD} модулем & \textbf{1.7} & -- & -- & 4.0\\		
			Обобщенное ядро с модулем анализа текста и ссылок & \textbf{2.0} & \textbf{8.5} & \textbf{11.5} & 4.0\\		
			\bottomrule %%% нижняя линейка
		\end{tabulary}%
	\end{SingleSpace}
\end{table}

\begin{table} [htbp]%
	\centering
	\caption{Средняя производительность обработки информации Веб-краулерами в российском сегменте Веб-пространства и время их настройки.}%
	\label{tab:crawlersRussian}% label всегда желательно идти после caption
	\renewcommand{\arraystretch}{1.5}%% Увеличение расстояния между рядами, для улучшения восприятия.
	\begin{SingleSpace}
		\begin{tabulary}{\textwidth}{@{}>{\zz}L >{\zz}C >{\zz}C >{\zz}C >{\zz}C@{}}%Вертикальные полосы не используются принципиально, как и лишние горизонтальные (допускается по ГОСТ 2.105 пункт 4.4.5) % @{} позволяет прижиматься к краям
			\toprule     %%% верхняя линейка
			Веб-краулеры & Время загрузки 2000 страниц (мин.) & Время загрузки 30000 страниц (мин.) & Время загрузки 50000 страниц (мин.) & Время настройки Веб-краулера (мин.) \\
			\midrule %%% тонкий разделитель. Отделяет названия столбцов. Обязателен по ГОСТ 2.105 пункт 4.4.5
			Heritrix & 5.0 & 12.9 & 22.3 & 20.0 \\				
			OpenWebSpider & 5.5 & 13.0 & 25.0 & 34.0 \\
			Methanol & 4.2 & 13.5 & 26.5 & 30.0 \\			
			Обобщенное ядро с \textit{Module\_BDD} модулем & -- & -- & \textbf{10.5} & 4.0\\
			Обобщенное ядро с \textit{Module\_MDD} модулем & -- & \textbf{6.4} & -- & 4.0\\			
			Обобщенное ядро с \textit{Module\_SDD} модулем & \textbf{2.0} & -- & -- & 4.0\\		
			Обобщенное ядро с модулем анализа текста и ссылок & \textbf{3.5} & \textbf{9.0} & \textbf{13.0} & 4.0\\		
			\bottomrule %%% нижняя линейка
		\end{tabulary}%
	\end{SingleSpace}
\end{table}

\begin{itemize}
	\item получено небольшое время добавления внешних модулей (алгоритмы обхода Веб-пространства, синтаксические анализаторы, методы обновления индекса и др.) к обобщенному ядру и их конфигурирования (табл.~\cref{tab:webCrawlerComparison},~\cref{tab:crawlersEnglish}) по сравнению с зарубежными аналогами, архитектура которых не предусматривает возможность масштабирования и добавления в нее других модулей и время настройки в среднем занимает \textit{28 минут}.
	\item Веб-краулер с обобщенным ядром имеет возможность добавления в него специализированных модулей и их адаптации для тематического поиска (рис.~\cref{fig:crawlerModes}): классический (рекурсивно обходит все встречающиеся гиперссылки); интересо-ориентированный; ориентированный на популярность; ориентированный на интересы и популярность. В исследованиях \cite{BlekanovBondarenko1,BlekanovBondarenko2} проверялась качество поиска и полезность этих модулей, а также их интеграция с обобщенным ядром. В то время как Heritrix, OpenWebSpider и Methanol Web Crawler используют только классический режим обхода Веб-пространства и не обрабатывают текст источников информации в Вебе, но поддерживают возможность поиска музыкальных, видео файлов, файлов PDF.
	
	\begin{figure}[ht]
		\centerfloat{
			\includegraphics[scale=0.6]{crawlerModes}
		}
		\caption{Режимы поиска Веб-краулера с обобщенным ядром.}\label{fig:crawlerModes}
	\end{figure}
	
	\item Веб-краулеры на основе обобщенного ядра с разными модулями загрузки Веб-страниц (\textit{Module\_BDD}, \textit{Module\_MDD} и \textit{Module\_SDD}) в классическом режиме (обрабатывая информацию только о ссылках) показали высокую производительность среди зарубежных аналогов (табл.~\cref{tab:crawlersEnglish},~\cref{tab:crawlersRussian}), как в российском, так и зарубежном сегменте Веба. Вместе с этим поисковый робот на основе обобщенного ядра с модулем анализа текста не уступает по производительности зарубежным аналогам, которые не обрабатывают текстовую информацию Веб-страниц, в англоязычном сегменте Веба, а в российском Вебе зарубежные реализации Веб-краулеров сильно уступают в производительности. 
	\item зарубежные аналоги показали низкую производительность обработки информационных Веб-ресурсов в российском сегменте Веба (табл.~\cref{tab:crawlersEnglish}), а \textit{70\%} ресурсов из этого сегмента Веб-краулеры Heritrix, OpenWebSpider и Methanol Web Crawler не смогли загрузить и обработать. Данный факт объясняется низкой приспособленностью зарубежных аналогов к российскому Вебу.
	\item Среди зарубежных аналогов неплохую эффективность загрузки и обработки \textit{2000}, \textit{300000} и \textit{500000} Веб-страниц показал Веб-краулер Heritrix. Его можно использовать как неплохой инструмент для поиска (с классическим алгоритмом обхода Веба) разнородной информации и добавления ее в индекс.
\end{itemize}

Таким образом, из поставленного эксперимента следует, что Веб-краулер с обобщенным ядром эффективней собирает и обрабатывает информацию с Веб-ресурсов, чем зарубежные аналоги, которые сильно уступают по производительности, гибкости и масштабируемости архитектуры, а также приспособленностью к обработке информации в российском сегменте Веба.

В дальнейшем созданный Веб-краулер с обобщенным ядром планируется использовать в прикладных исследованиях по повышению вебометрических рейтингов университетов, научных институтов и других научных организаций.

\subsection{Веб-краулер как инструмент для вебометрических исследований на примере анализа Веб-пространства СПбГУ}\label{subsec:ch1/sec3/sub2}

\paragraph{Введение.} За последние несколько лет в области информационного Веб-поиска все чаще появляются задачи, связанные с развивающимся научным направлением вебометрика (webometrics) \cite{HolmbergThelwall,Pechnikov,PechnikovChirkovChuiko,BlekanovSergeevPechnikov}. К актуальным направлениям вебометрических исследований относятся задачи анализа и выявления гиперссылочных структур различных сегментов Веб-пространства (например, академический сегмент Веба, университетский, и др.), решение которых влияет на качество присутствия этих сегментов в Вебе, на результаты ранжирования поисковых машин (Google, Yandex и др.) или, в случае университетского Веба, на вебометрический рейтинг (http://www.webometrics.info) различных университетов мира \cite{BlekanovSergeevPechnikov}.

 Для получения и обработки больших объемов информации о веб-сайтах и их гиперссылках используются Веб-краулеры (поисковые роботы), общей задачей которых является специализированный обход Веба с целью сбора информации или определения гиперссылочной структуры и полезности каких-либо информационных ресурсов.
 
\paragraph{Эксперимент.} В эксперименте ставилась задача анализа и выявления гиперссылочной структуры Веб-пространства Санкт-Петербургского государственного университета (СПбГУ). 
 
 Для эксперимента использовался программный комплекс обобщенного ядра поискового робота, который обладает высокой гибкостью и масштабируемостью в сравнении с зарубежными аналогами, сильно уступающими в производительности собора и обработки веб-ресурсов и имеющими слабую приспособленность к анализу российского сегмента Веба \cite{BlekanovSergeevMartynenko}. 
 
 К Веб-краулеру с обобщенным ядром дополнительно был разработан и добавлен специализированный алгоритм обхода веб-страниц, который собирает и обрабатывает только страницы из Веб-пространства СПбГУ. В свою очередь пространство СПбГУ состоит из веб-сайта главного домена и сайтов всех его поддоменов (рис.~\cref{fig:spbuWebSpace}).
 
\begin{figure}[ht]
    \centerfloat{
	        \includegraphics[scale=0.5]{spbuWebSpace}
	    }
    \caption{Веб-пространство СПбГУ.}\label{fig:spbuWebSpace}
\end{figure}

Используя программный комплекс на основе обобщенного ядра поискового робота со специализированным алгоритмом, запущенного с начального множества веб-страниц, требовалось в автоматизированном режиме получить значения следующих показателей, характеризующих гиперссылочную структуру Веб-пространства СПбГУ:
\begin{itemize}
	\item объем Веб-пространства СПбГУ (количество всех различных веб-страниц из Веб-пространства СПбГУ);
	\item количество всех поддоменов из Веб-пространства СПбГУ; 
	\item количество тупиковых (не имеющих ссылок) веб-страниц; 
	\item количество неработающих гиперссылок; 
	\item количество гиперссылок на внешние веб-ресурсы; 
	\item количество поддоменов, связанных с «Главной страницей» главного домена; 
	\item количество поддоменов, несвязанных с «Главной страницей» главного домена; 
	\item количество страниц, имеющие гиперссылки на «Главную страницу» главного домена; 
	\item количество страниц, не имеющие гиперссылки на «Главную страницу» главного домена; 
	\item гиперссылочная структура Веб-пространства СПбГУ в виде матрицы смежности. 
\end{itemize}

В качестве начального множества веб-страниц, с которого Веб-краулер запускал процесс сбора и обработки веб-ресурсов, брался URL-адрес главного веб-сайта СПбГУ -- «http://www.spbu.ru/».

\paragraph{Результаты эксперимента.} В ходе эксперимента всего Веб-краулером было обработано и проанализировано \textit{6\space429\space963} гиперссылки, которые содержались на страницах Веб-пространства СПбГУ. Из них: объем ссылок на внешние источники информации равен \textit{507\space168}, а объем внутренних ссылок (на страницы главного домена сайта СПбГУ и его поддоменов) -- \textit{5\space922\space795}. Кроме того, были получены следующие результаты (табл.~\cref{tab:indicators}):

\begin{table} [htbp]%
	\centering
	\caption{}%
	\label{tab:indicators}% label всегда желательно идти после caption
	\renewcommand{\arraystretch}{1.5}%% Увеличение расстояния между рядами, для улучшения восприятия.
	\begin{SingleSpace}
		\begin{tabulary}{\textwidth}{@{}>{\zz}L >{\zz}C@{}} %Вертикальные полосы не используются принципиально, как и лишние горизонтальные (допускается по ГОСТ 2.105 пункт 4.4.5) % @{} позволяет прижиматься к краям
			\toprule     %%% верхняя линейка
			Показатель & Значение показателя  \\
			\midrule %%% тонкий разделитель. Отделяет названия столбцов. Обязателен по ГОСТ 2.105 пункт 4.4.5
			объем Веб-пространства СПбГУ & 71688 \\				
			количество всех поддоменов из Веб-пространства СПбГУ & 315 \\
			количество поддоменов, сильно-связанных с «Главной страницей» главного домена & 280 \\
			количество поддоменов, несвязанных с «Главной страницей» главного домена & 35 \\
			количество тупиковых веб-страниц & 12516 \\
			количество неработающих гиперссылок & 680 \\
			количество страниц, имеющие гиперссылки на «Главную страницу» главного домена & 10632 \\
			количество страниц, не имеющие гиперссылки на «Главную страницу» главного домена & 61056 \\
			\bottomrule %%% нижняя линейка
		\end{tabulary}%
	\end{SingleSpace}
\end{table}

Также стоит отметить, что для простоты исследований и получения численных значений показателей было предложено хранить выявленную гиперссылочную структуру Веб-пространства СПбГУ в виде матрицы смежности A размерностью [71688\(\times\)71688]. С помощью нее и ее транспонированной формы удобно определять связанность между страницами и в дальнейшем оптимизировать структуру ссылок. 

\paragraph{Общие выводы.} По результатам эксперимента видно, что \textit{17,5\%} веб-страниц из всех страниц Веб-пространства СПбГУ занимают тупиковые страницы, которые, например, не учитывает поисковая машина Google в своем алгоритме ранжирования PageRank, что негативно влияет на коммуникабельность всего Веб-пространство университета в целом. 

Также из эксперимента наблюдается плохая связанность страниц Веб-пространства СПбГУ и его поддоменов со страницами главного сайта университета: только \textit{14,8\%} всех веб-страниц имеет ссылки на страницы главного сайта, а \textit{35} доменов со всеми своими страницами и вовсе никак не связаны с ним. 

Стоит отметить и большое количество неработающих гиперссылок (\textit{680} уникальных гиперссылок), которые неоднократно повторяются по всему Веб-пространству СПбГУ, тем самым снижают доверие пользователей к ресурсам сайтов университета. 

Таким образом, проведенный эксперимент демонстрирует слабую связанность и коммуникабельность внутренних ресурсов Веб-пространства СПбГУ, что влечет и его слабую позицию в вебометрическом рейтинге сайтов университетов мира.

\subsection{Вебометрические исследования сегмента университетского Веба с помощью поискового робота}\label{subsec:ch1/sec3/sub3}

\paragraph{Введение.} За последние годы все больше российских ВУЗов следят за позициями своих веб-сайтов в вебометрическом рейтинге испанской группы Cybermetrics Lab \cite{RankingWeb}, который входит в пятерку мировых рейтингов наряду с такими рейтингами, как Times, QS, шанхайский и т. п. Этот рейтинг оценивает лишь качество сайта университета, но качество веб-сайта оказывает большое влияние на «видимость» университета в целом. Ведь по публикациям ученых видны только они, но не виден университет.

Для повышения рейтинга университетского сайта необходимо работать в таких направлениях, как увеличение объема, улучшение качества контента, увеличение количества гиперссылок на сайт с авторитетных веб-сайтов, а также оптимизация внутренней гиперссылочной структуры сайта. Исследованиям в этой области посвящены работы \cite{HolmbergThelwall,Pechnikov,Pechnikov2,PechnikovChirkovChuiko,BlekanovSergeevPechnikov,BlekanovSergeevMaksimov,BlekanovSergeev}. Мы же решили сосредоточиться на вопросах внутренней гиперссылочной структуры как целого Веб-пространства университета, так и отдельных сегментов, из которых оно состоит. Правильно организованная веб-структура может значительно увеличить коммуникабельность сайта и повысить его показатель цитируемости в поисковых системах, увеличить количество внешних ссылок, улучшить позиции университетского веб-сайта в вебометрическом рейтинге.

\paragraph{Эксперимент.} В эксперименте ставилась задача анализа и выявления гиперссылочной структуры одного из сегментов, составляющих Веб-пространство СПбГУ, -- веб-сайт факультета ПМ–ПУ, а также получения рекомендаций по улучшению этой структуры.

В качестве инструмента для достижения поставленной в эксперименте цели использовалась аналитическая система на основе специ- ализированного поискового робота, которая была успешно апробирована на исследованиях зарубежного и российского сегментов Веба \cite{BlekanovSergeevPechnikov,BlekanovSergeevMaksimov,BlekanovSergeev,BlekanovSergeevMartynenko}.

Специализация используемого поискового робота была сконфигурирована таким образом, чтобы робот собирал и обрабатывал только страницы в рамках сегмента Веба, содержащего веб-сайт факультета ПМ–ПУ СПбГУ, и гиперссылки с этого сайта на внешние источники информации (рис.~\cref{fig:spbuApmathWebSpace}).

\begin{figure}[ht]
	\centerfloat{
		\includegraphics[scale=1]{spbuApmathWebSpace}
	}
	\caption{Веб-пространство СПбГУ и включенный в него исследуемый сегмент Веба.}\label{fig:spbuApmathWebSpace}
\end{figure}

С помощью аналитической системы на основе специализированного поискового робота, запущенной с начальным URL-адресом http://www.apmath.spbu.ru/, требовалось выявить гиперссылочную структуру сайта ПМ–ПУ в виде матрицы смежности и определить значения параметров, приведенных в таблице~\cref{tab:researchedParameters}.

\begin{table} [htbp]%
	\centering
	\caption{Исследуемые параметры.}%
	\label{tab:researchedParameters}% label всегда желательно идти после caption
	\renewcommand{\arraystretch}{1.5}%% Увеличение расстояния между рядами, для улучшения восприятия.
	\begin{SingleSpace}
		\begin{tabulary}{\textwidth}{@{}>{\zz}L >{\zz}C@{}} %Вертикальные полосы не используются принципиально, как и лишние горизонтальные (допускается по ГОСТ 2.105 пункт 4.4.5) % @{} позволяет прижиматься к краям
			\toprule     %%% верхняя линейка
			ID параметра & Параметр  \\
			\midrule %%% тонкий разделитель. Отделяет названия столбцов. Обязателен по ГОСТ 2.105 пункт 4.4.5
			1 & Общее количество уникальных внутренних страниц сайта ПМ–ПУ \\				
			2 & Количество страниц, связанных с «Главной страницей» сайта ПМ–ПУ \\
			3 & Количество страниц, не связанных с «Главной страницей» сайта ПМ–ПУ \\
			4 & Количество тупиковых веб-страниц \\
			5 & Количество уникальных гиперссылок сайта ПМ–ПУ \\
			6 & Количество неработающих гиперссылок0 \\
			7 & Количество гиперссылок на внешние веб-ресурсы \\
			8 & Количество внутренних гиперссылок на документы форматов *.doc (*.docx), *.pdf, *excel, *.ppt \\
			9 & Количество гиперссылок с переадресацией \\
			\bottomrule %%% нижняя линейка
		\end{tabulary}%
	\end{SingleSpace}
\end{table}

\paragraph{Результаты эксперимента.} В ходе эксперимента поисковым роботом на веб-сайте факультета ПМ–ПУ была обработана и проанализирована \textit{2 258 521} гиперссылка. Из них \textit{2 026 285} гиперссылок на внутренние веб-страницы сайта ПМ–ПУ, \textit{119 931} ссылка на страницы из Веб-пространства СПбГУ и \textit{112 305} ссылок на внешние источники информации. Кроме того, с помощью построенной матрицы смежности и ее транспонированной формы для параметров из таблицы~\cref{tab:researchedParameters} были получены числовые значения в виде гистограмм и круговых диаграмм (рис.~\cref{fig:histogramOfTab1},~\cref{fig:linkDiagram}).

\begin{figure}[ht]
	\centerfloat{
		\includegraphics[scale=0.5]{histogramOfTab1}
	}
	\caption{Гистограммы значений показателей таблицы~\cref{tab:researchedParameters}.}\label{fig:histogramOfTab1}
\end{figure}

\begin{figure}[ht]
	\centerfloat{
		\includegraphics[scale=0.5]{linkDiagram}
	}
	\caption{Круговая диаграмма внешних и внутренних ссылок.}\label{fig:linkDiagram}
\end{figure}

\paragraph{Общие выводы.} По результатам эксперимента видно, что тупиковые страницы (не имеющие никаких других ссылок), не являющиеся медиа-контентом, отсутствуют. В свою очередь, 16,2\% страниц не связаны с главной веб-страницей главного сайта. Однако если отбросить страницы с документами, то это отношение составит всего около 1\%.
Несмотря на относительно неплохую связность страниц сайта и отсутствие тупиковых страниц, ситуацию ухудшает огромное число гиперссылок, переадресующих пользователя в основном на главную страницу сайта или ее иностранные аналоги. Кроме того, не все гиперссылки приводятся к единой нормальной форме на стороне сервера. Большое количество таких гиперссылок влекут вероятность некорректного цитирования информационных ресурсов веб-сайта ПМ–ПУ в поисковых системах, а также оказывают негативное влияние на вебометрический рейтинг университета в целом.
Стоит отметить и большое количество веб-страниц, при переходе на которые возникает ошибка, вызванная отсутствием предполагаемых данных в указанном месте сервера или неверной обработкой адреса. Наличие таких страниц снижает доверие пользователей к ресурсам сайта факультета.
Таким образом, для улучшения присутствия веб-сайта факультета ПМ–ПУ в Вебе и улучшения его показателя цитируемости в поисковых системах необходимо разрешить выявленные в эксперименте проблемы на практике.

\section{Анализ гиперссылочной структуры ресурсов}\label{sec:ch1/sec4}

\subsection{Исследование закономерностей в гиперссылочной структуре сайтов большой размерности (должна быть на англ.)}\label{subsec:ch1/sec4/sub5}

\paragraph{Актуальность.} В настоящее время редкая организация не имеет собственного сайта. Сайт, в определенном смысле, лицо организации и его качеству придается большое значение. Качество сайта складывается из многих составляющих. Это и количество страниц, и их дизайн, и содержание страниц, и структура сайта.

Представление структуры сайта в виде ориентированного графа, узлы которого -- документы, а дуги -- ссылки, является общепринятым \cite{BroderKumarMaghoul}. Глобальными структурными характеристиками информационных сетей в Вебе занимаются многие исследователи. Так, в работе Broder A. и Kumar F. \cite{BroderKumarMaghoul} крупные сообщества сайтов представляются в ввиде графовой компоненты сильной связности, компонент In, Out и Tubes; в работах \cite{Thelwall,ThelwallZuccala,ThelwallWilkinsonMusgrove,PechnikovNwohiri} рассматривается распределение внешних ссылок и индекс цитирования различных университетских сайтов; в \cite{BlekanovSergeevMaksimovBOWTIE} вводятся характеристики связности сайта; в \cite{KenekayoroBuckleyThelwall} исследуется метод автоматической классификации ссылок и страниц по их характеристикам; и др.

Цель настоящей работы – исследование структуры сайта. Точнее – установления вида функциональной зависимости, связывающей число страниц сайта с числом его внутренних ссылок.

\subsubsection{Проверка гипотезы}

\paragraph{Постановка задачи.} Для решения этой задачи использовался специальный поисковый робот \cite{BlekanovSergeevMartynenko}. В процессе обхода сайта робот составляет два списка: список найденных страниц (узлов веб-графа) и список ссылок, связывающих найденные страницы (дуг веб-графа).

Шагом алгоритма просмотра сайта будем считать акт нахождения \(e\) ссылок. То есть, в результате \(i\) шагов находятся \(E_i = ei\) ссылок. Количество найденных на \(i\) шагах страниц обозначим через \(v_i = v(E_i)\). Будем обозначать число всех ссылок сайта через \(e_0\), число всех страниц -- \(v(e_0) = v_0\) . Поисковый робот позволяет не только найти число страниц и ссылок сайта, но и получить график функции \(v(e)\)).

Очевидно, что \(v \leq e \). Причем \(v_0 < e_0\) (случай \(v_0 = e_0\) для сайтов -- экзотический). Очевидно, также, что рост \(v(e)\) постепенно замедляется. Это объясняется тем, что с ростом числа найденных страниц растет вероятность того, что очередная ссылка указывает на уже найденную страницу. 

Попытаемся подобрать аналитическую функцию, приближенно описывающую экспериментальную \(v(E)\), со следующими свойствами:
\begin{enumerate}
	\item На плоскости \((e, v)\) проходит через начало координат.
	\item Возрастает с ростом \(e\).
	\item Производная функции убывает.
	\item Функция имеет простой вид и зависит от небольшого числа параметров.
\end{enumerate}

\paragraph{Эксперимент.} Будем сопоставлять функцию с экспериментальным набором \(V_i, E_i (i = 1,2, \dots, N)\), где \(N = \lceil \frac{e_0}{e} \rceil\). Нами исследованы сайты четырех университетов и получены соответствующие наборы (табл.~\cref{tab:uniSites}).

\begin{table} [htbp]%
	\centering
	\caption{Сайты исследуемых университетов.}%
	\label{tab:uniSites}% label всегда желательно идти после caption
	\renewcommand{\arraystretch}{1.5}%% Увеличение расстояния между рядами, для улучшения восприятия.
	\def\tabularxcolumn#1{m{#1}}
	\begin{tabularx}{\textwidth}{@{}>{\raggedright}X >{\centering}m{3.5cm} >{\centering}m{2.5cm} >{\centering\arraybackslash}m{2.5cm}@{}}% Вертикальные полосы не используются принципиально, как и лишние горизонтальные (допускается по ГОСТ 2.105 пункт 4.4.5) % @{} позволяет прижиматься к краям
		\toprule     %%% верхняя линейка
		Университет & URL-адрес & Всего страниц & Всего ссылок \\
		\midrule %%% тонкий разделитель. Отделяет названия столбцов. Обязателен по ГОСТ 2.105 пункт 4.4.5
		%Санкт-Петербургский государственный университет & www.spbu.ru & 41 183 & 2 664 000 \\
		Московский государственный университет & www.msu.ru & 47 832 & 1 891 000 \\				
		Университет Айдзу & www.u-aizu.ac.jp & 4 161 & 49 900 \\
		Токийский университет & www.u-tokyo.ac.jp & > 17 000 & 240 000 \\			
		\bottomrule %%% нижняя линейка
	\end{tabularx}%
\end{table}

Для оценки качества приближения будем использовать среднюю относительную погрешность:
\[
\Delta = \frac{1}{N} \sum \frac{\lvert V_i - v_i \rvert}{v_i}
\]

Далее рассматриваются три варианта аппроксимирующей функции:
\[v^{(1)} = \alpha E^\beta ,\]
\[v^{(2)} = \frac{\alpha E}{E + \beta},\]
\[v^{(3)} = \alpha (\ln(1 + E))^\beta .\]

Все три функции удовлетворяют требованиям 1, 2 и 3 при \(\alpha > 0, \beta > 0\). \(v^{(1)}\) удовлетворяет требованию 3, если к тому же  \(\beta < 0\).

Рассмотрим их последовательно:

\begin{enumerate}
	\item Прологарифмируем уравнение: \[\ln v^{(1)} = \ln \alpha + \beta \ln E .\] Получаем систему линейных уравнений \[x + A_i y = B_i, i = \overline{1,N} ,\] где \(x = \ln \alpha, y = \beta, A_i = \ln E_i, B_i = \ln V_i\). Ее решение методом наименьших квадратов: \[\alpha = \exp \frac{C_1C_4 - C_2C_3}{N(C_1^2 - C_3)}, \beta = \frac{C_1C_2 - C_4}{C_1^2 - C_3},\] где \(C_1 = \sum_{1}^{N}A_i, C_2 = \sum_{1}^{N}B_i, C_3 = \sum_{1}^{N}A_i^2, C_4 = \sum_{1}^{N}A_i B_i\).
	\item Система линейных уравнений: \[\alpha E_i - \beta V_i = V_i E_i, i = \overline{1,N}.\] Ее решение методом наименьших квадратов: \[\alpha = \frac{a_3 a_4 - a_2 a_5}{a_1 a_4 - a_2^2}, \beta = \frac{a_2 a_3 = a_1 a_5}{a_1 a_4 - a_2^2}, \] где \(a_1 = \sum_{1}^{N} E_i^2, a_2 = \sum_{1}^{N} E_i V_i, a_3 = \sum_{1}^{N} E_i^2 V_i, a_4 = \sum_{1}^{N} V_i^2, a_5 = \sum_{1}^{N} V_i^2 E_i\).
	\item Логарифмируем \(v^{(3)}\): \[\ln v^{(3)} = \ln \alpha + \beta \ln \ln (1 + E).\] Получаем систему, почти совпадающую со случаем 1). Ее решение будет таким же, за исключением того, что в первом случае \(A_i = \ln E_i\), а в этом \[A_i = \ln \ln (1+E).\]
\end{enumerate}

\paragraph{Результаты эксперимента.} На рисунках~\cref{fig:msuFunc},~\cref{fig:aizuFunc},~\cref{fig:tuniFunc} представлены графики функций \(v^{(1)}, v^{(2)}, v^{(3)}\) и \(V\) для каждого из перечисленных университетов.

\begin{figure}[ht]
	\centerfloat{
		\includegraphics[scale=0.35]{msuFunc}
	}
	\caption{Функции \(v^{(1)}, v^{(2)}, v^{(3)}\) и \(V\) Московского государственного университета.}\label{fig:msuFunc}
\end{figure}

\begin{figure}[ht]
	\centerfloat{
		\includegraphics[scale=0.35]{aizuFunc}
	}
	\caption{Функции \(v^{(1)}, v^{(2)}, v^{(3)}\) и \(V\) для университета Айдзу.}\label{fig:aizuFunc}
\end{figure}

\begin{figure}[ht]
	\centerfloat{
		\includegraphics[scale=0.35]{tuniFunc}
	}
	\caption{Функции \(v^{(1)}, v^{(2)}, v^{(3)}\) и \(V\) для Токийского университета.}\label{fig:tuniFunc}
\end{figure}

В табл.~\cref{tab:uniErrors} приведены относительные погрешности каждой из формул для каждого из университетов.

\begin{table} [htbp]%
	\centering
	\caption{Средняя относительная погрешность функций \(v^{(1)}, v^{(2)}, v^{(3)}\) для каждого университета.}%
	\label{tab:uniErrors}% label всегда желательно идти после caption
	\renewcommand{\arraystretch}{1.5}%% Увеличение расстояния между рядами, для улучшения восприятия.
	\def\tabularxcolumn#1{m{#1}}
	\begin{tabularx}{\textwidth}{@{}>{\raggedright}X >{\centering}m{3.5cm} >{\centering}m{2.5cm} >{\centering\arraybackslash}m{2.5cm}@{}}% Вертикальные полосы не используются принципиально, как и лишние горизонтальные (допускается по ГОСТ 2.105 пункт 4.4.5) % @{} позволяет прижиматься к краям
		\toprule     %%% верхняя линейка
		Университет & \(v^{(1)}\) & \(v^{(2)}\) & \(v^{(3)}\) \\
		\midrule %%% тонкий разделитель. Отделяет названия столбцов. Обязателен по ГОСТ 2.105 пункт 4.4.5
		%Санкт-Петербургский государственный университет & www.spbu.ru & 41 183 & 2 664 000 \\
		Московский государственный университет & 0,092 & 0,037 & 0,014 \\				
		Университет Айдзу & 0,167&  0,209 &  0,229 \\
		Токийский университет & 0,068 & 0,076 & 0,062 \\			
		\bottomrule %%% нижняя линейка
	\end{tabularx}%
\end{table}

\paragraph{Выводы.} Рассмотрена проблема нахождения функции, аппроксимирующей экспериментальный график зависимости числа найденных страниц сайта от числа найденных ссылок. Предложены три варианта аппроксимации. Выяснено, что наилучшие приближения получаются для степенной и дробно-линейной функции. Предложенные функции могут быть использованы для изучения параметров сайтов и их кластеризации.

\subsection{Применение модифицированного алгоритма LSH для кластеризации внешнего окружения веб-пространства университетов}\label{subsec:ch1/sec4/sub7}

Для большинства крупных организаций немаловажное значение имеет их рейтинг, который рассчитывается в зависимости от параметров, связанных с их родом деятельности. В частности, на общий рейтинг ВУЗа большое влияние оказывает его вебометрический рейтинг. Известно, одним из главных показателей, который влияет на вебометрический рейтинг любой организации, в том числе и университета, является количество внешних ресурсов \cite{RankingWeb}, ссылающихся на сайт университета. Но кроме количества ссылающихся ресурсов также важно понимать качество этих ресурсов, их природу, определить к какой области относится тот или иной внешний ресурс. Данные веб-ресурсы образуют группы сайтов с одинаковым родом деятельности. Таким образом, возникает задача выявления этих групп -- кластеризации. Требуется определить степень влияния количества и размеров найденных групп на вебометрический рейтинг сайтов университетов. По найденным кластерам можно определить, с какими группами внешних ресурсов следует сайтам университетов выстраивать гиперссылочные взаимосвязи для повышения цитируемости.

Объектом данного исследования являются университетские сайты, которые имеют чрезвычайно большие размеры (например, сайт СПбГУ содержит более \textit{50} тыс. внутренних веб-страниц и около 5 млн. гиперссылок) \cite{BlekanovMoskalets,BlekanovSergeevMaksimovBOWTIE}. Окружение таких сайтов может составлять десятки – сотни тысяч страниц. Стандартными методами кластеризации такого объема внешних веб-ресурсов не обойтись, так как при работе с большими коллекциями документов многие из данных методов показывают крайне неудовлетворительные результаты в плане производительности \cite{EneImMoseley} (например, метод Single linkage позволяет создавать кластеры произвольной формы, однако имеет высокую трудоемкость -- \(O(n^2)\), где \(n\) -- число документов). Для того чтобы избавиться от «проклятия» размерности применяется вероятностный метод понижения размерности многомерных данных Locality-Sensitive Hashing (в дальнейшем LSH), основная идея которого состоит в подборе хэш-функций для некоторых измерений для того, чтобы похожие объекты попадали в одну корзину \cite{Buhler}.

\subsubsection{Уменьшение размерности для анализа больших коллекций текстовых документов}

Для преобразования текстовых документов в числовые множества в данной работе использовался метод Shingling \cite{Broder}, который разбивает каждый документ на небольшие множества по \(k\) слов в каждом. Далее, для сравнения документов, применялась технология Min Hashing, которая позволяет быстро сравнивать множества, содержащие большое число элементов.

\textit{Мера Жаккара.} В работе для определения похожести двух текстов применяется мера Жаккара, которая похожесть множеств определяет отношением числа элементов, входящих в пересечение двух множеств к числу элементов, входящих в объединение этих множеств \cite{SingthongchaiNiwattanakul}:
\[
\textit{Sim}(C_1, C_2) = \frac{\lvert C_1 \cap C_2\rvert}{\lvert C_1 \cup C_2\rvert}
\]

Так как каждый документ преобразуется во множества, состоящие из \(k\) слов, то набор документов можно представить в виде сильно разреженной булевой матрицы, где столбцы представляют собой документы, а строки -- элементы универсального множества (например, множество элементов, где каждый элемент представляет собой множество \(k\) слов). Элемент такой матрицы равен единице, если документ (столбец) содержит данное множество k-слов. В противном случае элемент матрицы равен нулю.

\textit{Minhashing.} Для большой коллекции документов булева матрица будет сильно разрежена. Соответственно, матрица, описывающая такую коллекцию, займет много места в памяти. Кроме того, дальнейшая ее обработка так же займет большое количество времени. Для того чтобы решить возникшие проблемы, данную матрицу преобразуем в матрицу, хранящую определенное количество хэш-функций и информацию о сходстве между похожими документами.

\textit{Minhash}-функция \(h(C)\) -- это номер первой строки для столбца \(C\) в булевой матрице, где строки перемешаны случайным образом \cite{ChumPerdochMatas}.

Как видим, число случайных перестановок задает число Minhash-функций. Например, можно использовать сто случайных перестановок для создания ста сигнатур для каждого столбца матрицы.

Сигнатуры могут быть записаны в другой матрице сигнатур (Signature matrix), чьи колонки представляют собой документы, а строки -- Minhash-значения.

Чем больше хэш-функций, тем выше вероятность того, что \(\textit{Sim}(C_1, C_2) = \textit{Sim}(M[C_1], M[C_2])\), где \(C_i\) -- столбец булевой матрицы, \(M[C_i]\) -- столбец матрицы сигнатур. Это важное свойство позволяет преобразовывать булевы матрицы с большим количеством строк в небольшие матрицы сигнатур, сохраняя сходство похожих множеств. Соответственно, похожесть двух столбцов определяется долей строк, в которых они равны.

Таким образом, каждый документ может быть представлен в виде вектора, число элементов которого равно количеству Minhash-функций.

\textit{Locality-Sensitive Hashing (LSH).} Основная идея: сгенерировать из большого множества документов маленькие списки пар документов, чья схожесть должна быть посчитана. Для сравнения двух документов (столбцов) устанавливается порог \(t\) \((t < 1)\). Пара документов считается похожей только в том случае, если доля одинаковых значений в матрице сигнатур больше \(t\). Для матрицы сигнатур необходимо несколько раз вычислить хэш-значения, и поместить документы с одинаковым значением в одну корзину (\textit{bucket}). Документы, которые хоть раз попали в одну корзину, будут рассмотрены как кандидаты на сравнение \cite{GionisIndykMotwani}.

Для множественного подсчета хэш-функций столбцов, необходимо разбить матрицу сигнатур \textit{M} на \(b\) частей по \(r\) строк в каждой. Для каждого \(b\) подсчитать хэш-значения столбцов и поместить столбцы с равным значением в одну корзину. Кандидатами на сравнение будут те столбцы, которые хоть раз попали в одну корзину. Для правильной работы алгоритма необходимо настроить \(r\) и \(b\) таким образом, чтобы похожие документы попадали в одну корзину, а непохожие -- в разные.

\subsubsection{Иерархическая кластеризация с использованием LSH}

Данный алгоритм кластеризации использует хэш-таблицы, которые были сформированы в результате LSH \cite{KogaIshibashiWatanabe}. Данный алгоритм кластеризации с высокой долей вероятности создает такие же кластеры, как и в методе Single linkage. Ниже представлено детальное описание алгоритма.

\textit{Предварительные условия:}
\begin{itemize}
	\item \(t < 1\) -- порог, задающий минимальную схожесть документов;
	\item \(r = 1\) -- начальное значение строк матрицы сигнатур в каждой группе;
	\item \(r_\textit{min}\) -- минимальное значение строк матрицы сигнатур в каждой группе;
	\item \(\Delta\) -- коэффициент уменьшения параметра \(r\).
\end{itemize}

Каждый документ представляет собой отдельный кластер.

\textit{Шаг 1.} Для каждой группы \(b\) в каждом столбце вычислить хэш-функции, и сохранить столбцы, у которых хэш-значения хотя бы раз попали в одну корзину, при условии, что в корзине должны находиться столбцы, принадлежащие разным кластерам. Если, например, какие то два столбца принадлежат одному кластеру, то один из случайно выбранных столбцов удаляется из корзины.

\textit{Шаг 2.} Для каждого из столбцов, входящих в одну корзину, отобрать пары кластеров, расстояние между которыми больше \(t\).

\textit{Шаг 3.} Пары кластеров, соответствующие парам, полученным на шаге 2, объединяются в один.

\textit{Шаг 4.} Если \(r \leq r_\textit{min}\), то алгоритм прекращает работу. Иначе, переход на шаг 5.

\textit{Шаг 5.} \(r = r - \Delta\). Переход на шаг 1.

\textit{Оценка качества кластеризации на основе LSH.} Для оценки качества модифицированного метода агломеративной кластеризации с использованием LSH в данной работе использовалась тестовая коллекция текстовых документов Reuters-21578, содержащая \textit{21 578} документов. На данной коллекции оценивались два метода иерархической кластеризации: стандартный алгоритм иерархической кластеризации single-link и модифицированный алгоритм single-link, основанный на применении LSH. В качества основных метрик качества в работе использовались точность (\(R\)) полнота (\(P\)), аккуратность (\textit{Acc}) и \(F\)-мера.

В табл.~\cref{tab:clasterizationReuters} приведены результаты оценки качества иерархической кластеризации для каждого из алгоритмов, а так же время работы каждого алгоритма при кластеризации \textit{1000}, \textit{10 000} и \textit{20 000} документов.

\begin{table}[ht]%
	\caption{Результаты оценки качества кластеризации на коллекции Reuters.}%
	\label{tab:clasterizationReuters}% label всегда желательно идти после caption
	\renewcommand{\arraystretch}{1.6}%% Увеличение расстояния между рядами, для улучшения восприятия.
	\def\tabularxcolumn#1{m{#1}}
	\begin{tabularx}{\textwidth}{@{}>{\raggedright}X >{\centering}m{3.2cm} >{\centering}m{1.5cm} >{\centering}m{1.5cm} >{\centering}m{1.5cm} >{\centering}m{2.3cm} >{\centering\arraybackslash}m{2cm}@{}}% Вертикальные полосы не используются принципиально, как и лишние горизонтальные (допускается по ГОСТ 2.105 пункт 4.4.5) % @{} позволяет прижиматься к краям
		\toprule     %%% верхняя линейка
		Алгоритм & Количество документов & \textit{Acc}, \% & \(R\), \% & \(P\), \% & \(F\)-мера, \% & Время работы\\
		\midrule %%% тонкий разделитель. Отделяет названия столбцов. Обязателен по ГОСТ 2.105 пункт 4.4.5
		 & 1000 & 75 & 79 & 60 & 68 & 4 с \\
		 Single-Link & 10 000 & 79 & 82 & 62 & 71 & 410 с \\
		 & 20 000 & -- & -- & -- & -- & > 1 ч \\
		 \midrule
		 & 1000 & 72 & 81 & 66 & 73 & 3 с \\
		 Single-Link + LSH & 10 000 & 72 & 80 & 69 & 74 & 41 с \\
		 & 20 000 & 78 & 83 & 64 & 72 & 90 с \\
		\bottomrule %%% нижняя линейка
	\end{tabularx}%
\end{table}

Как видно из таблицы, аккуратность и полнота метода Single-Link + LSH близки по значениям к аккуратности и полноте метода Single-Link. Точность же разработанного метода иногда превосходит точность метода Single-Link. \(F\)-меры у обоих методов примерно одинаковые.

На рис.~\cref{fig:clasterizationTime} приведена зависимость продолжительности работы алгоритмов кластеризации от числа входных документов.

\begin{figure}[ht]
	\centerfloat{
		\includegraphics[scale=0.7]{clasterizationTime}
	}
	\caption{Время работы алгоритмов кластеризации.}\label{fig:clasterizationTime}
\end{figure}

Графики показываю, что алгоритм кластеризации с использованием LSH работает гораздо быстрее, чем алгоритм без использования LSH. С увеличением числа документов время работы алгоритма с использованием LSH линейно растет.

\subsubsection{Эксперимент. Исследование сайтов университетов методом кластеризации на основе LSH}
\paragraph{Постановка эксперимента.} В эксперименте требовалось для заданного списка сайтов университетов России, США и Великобритании, занимающих по своим регионам ведущие позиции в вебометрическом рейтинге \cite{RankingWeb}, с помощью специализированного поискового робота \cite{BlekanovSergeevMartynenko} и базы данных Majestic \cite{Majestic} получить списки и содержимое всех внешних веб-страниц, которые их цитируют. А также с помощью разработанного авторами метода агломеративной кластеризации на основе алгоритма LSH выявить целевые группы найденных внешних веб-ресурсов с одинаковым родом деятельности и степень их влияния на вебометрический рейтинг сайтов исследуемых ВУЗов.

Для исследования выбраны следующие сайты ВУЗов:
\begin{itemize}
	\item Московский государственный университет им. М.В. Ломоносова (msu.ru), Россия;
	\item Санкт-Петербургский государственный университет (spbu.ru), Россия;
	\item Новосибирский государственный университет (nsu.ru), Россия;
	\item Массачусетский технологический институт (mit.edu), США;
	\item Гарвардский университет (harvard.edu), США;
	\item Стэнфордский университет (stanford.edu), США;
	\item Кембриджский университет (cam.ac.uk), Великобритания;
	\item Оксфордский университет (ox.ac.uk), Великобритания;
	\item Университетский колледж Лондона (ucl.ac.uk), Великобритания.
\end{itemize}

\paragraph{Результаты сбора.} В итоге сбора данных для анализа внешних веб-ресурсов сайтов исследуемых университетов были получены следующие результаты, представленные в табл.~\cref{tab:externalResources}.

\begin{table}[ht]%
	\caption{Количество внешних ресурсов, окружающих университеты.}%
	\label{tab:externalResources}% label всегда желательно идти после caption
	\renewcommand{\arraystretch}{1.6}%% Увеличение расстояния между рядами, для улучшения восприятия.
	\def\tabularxcolumn#1{m{#1}}
	\begin{tabularx}{\textwidth}{@{}>{\centering}X  >{\centering}m{2.6cm} >{\centering}m{2.6cm} >{\centering}m{2.6cm} >{\centering}m{2.6cm} >{\centering\arraybackslash}m{2.6cm}@{}}% Вертикальные полосы не используются принципиально, как и лишние горизонтальные (допускается по ГОСТ 2.105 пункт 4.4.5) % @{} позволяет прижиматься к краям
		\toprule     %%% верхняя линейка
		URL-адрес ВУЗа & Позиция в рейтинге Webometrics &  Внешние ссылки на сайте ВУЗа & Внешние домены на сайте ВУЗа & Количество цитирующих сайт ВУЗа веб-страниц & Количество цитирующих сайт ВУЗа доменов \\
		\midrule %%% тонкий разделитель. Отделяет названия столбцов. Обязателен по ГОСТ 2.105 пункт 4.4.5
		spbu.ru & 539 & 3 280 & 469 & 1 599 059 & 16 028 \\
		msu.ru & 129 & 817 & 262 & 7 039 127 & 39 416 \\
		nsu.ru & 616 & 5 747 & 921 & 926 004 & 15 186 \\
		harvard.edu & 1 & 918 & 163 & 75 994 723 & 319 445 \\
		stanford.edu & 2 & 175 & 79 & 29 551 130 & 311 148 \\
		mit.edu & 3 & 9 & 6 & 41 271 678 & 324 989 \\
		ox.ac.uk & 16 & 8 075 & 1 631 & 8 920 524 & 117 959 \\
		cam.ac.uk & 15 & 3 385 & 1 174 & 12 084 107 & 120 796 \\
		ucl.ac.uk & 24 & 25 218 & 5 683 & 4 733 566 & 66 035 \\
		\bottomrule %%% нижняя линейка
	\end{tabularx}%
\end{table}

Таблица показывает, что университеты США, занимающие первые три позиции вебометрического рейтинга, имеют значительно больше внешних ресурсов, ссылающихся на них, чем ведущие университеты России и Великобритании. Ведущие университеты Великобритании также опережают российские по общему количеству ссылок и доменов, ссылающихся на них. Количество внешних ресурсов напрямую влияет на такой вебометрический индикатор, как Impact ресурса в Вебе.

На показатель видимости ресурса в Вебе, помимо числа внешних ссылающихся ресурсов, также влияет и их качество. Для кластеризации внешних Веб-ресурсов были выбраны только англоязычные и русскоязычные страницы Веба. В табл.~\cref{tab:externalResourcesClasterization} приведено количество доменов для каждого университета, к которым применялся метод агломеративной кластеризации с использованием LSH.

\begin{table}[ht]%
	\centering
	\caption{Количество внешних ресурсов для кластеризации.}%
	\label{tab:externalResourcesClasterization}% label всегда желательно идти после caption
		\begin{tabular}{| c | c | c | c | c |}% Вертикальные полосы не используются принципиально, как и лишние горизонтальные (допускается по ГОСТ 2.105 пункт 4.4.5) % @{} позволяет прижиматься к краям
			\hline
			\makecell{\\Университет} & \multicolumn{2}{c|}{\makecell{Внешние домены на \\ сайте ВУЗа}} &  \multicolumn{2}{c|}{\makecell{Цитирующие сайт ВУЗа \\ домены}}\\
			\cline{2-5}
			& на русском & на английском & на русском & на английском\\
			\hline
			spbu.ru & 342 & 81 & 7 373 & 3 929  \\
			\hline
			msu.ru & 177 & 62 & 3 343 & 1 008 \\
			\hline
			nsu.ru & 535 & 242 & 2 508 & 1 428  \\
			\hline
			harvard.edu & 4 & 151 & 1 545 & 16 632  \\
			\hline
			stanford.edu & 5 & 65 & 409 & 6 088  \\
			\hline
			mit.edu & 3 & 3 & 1 625 & 24 930  \\
			\hline
			ox.ac.uk & 29 & 1 399 & 126 & 3 187  \\
			\hline
			cam.ac.uk & 91 & 4 668 & 671 & 9 196 \\
			\hline
			ucl.ac.uk & 14 & 1 054 & 162 & 3 112  \\
			\hline
		\end{tabular}%
\end{table}

Ниже приведены результаты кластеризации внешних ресурсов. Для определения их тематики извлекались наиболее часто встречающиеся множества слов (shingles), которые присутствовали в кластере. В результате анализа были выделены основные часто встречающиеся группы: университеты, научные сообщества, поисковые системы, социальные сети (включая различные блоги, форумы и т.д.), медиасфера, новостные порталы и сайты других организаций. Таким образом, определялась численность каждой группы для университета.

\paragraph{Результаты кластеризации внешних доменов на сайте ВУЗа.} Внешние домены -- это домены, на которые ссылаются исследуемые сайты университетов. Ввиду ограниченного числа внешних ссылок на сайтах университетов США, были рассмотрены только внешние ресурсы университетов России и Великобритании. Были получены кластеры для каждого из исследуемых университетов. Ниже представлена сравнительная гистограмма внешнего окружения университетов России и Великобритании (рис.~\cref{fig:histogramUKRU}).

\begin{figure}[ht]
	\centerfloat{
		\includegraphics[scale=0.7]{histogramUKRU}
	}
	\caption{Сравнительная гистограмма университетов России и Великобритании.}\label{fig:histogramUKRU}
\end{figure}

Рисунок показывает, что внешнее ресурсы, на которые ссылаются сайты университетов, практически совпадают. Университеты и научные сообщества составляют значительную долю внешних ресурсов, на которые ссылаются сайты университетов России и Великобритании.

\paragraph{Результаты кластеризации цитирующих сайт ВУЗа доменов.} Цитирующие веб-ресурсы -- это сайты, сгруппированные по доменам, которые ссылаются на исследуемые сайты университетов. Были получены кластеры для каждого из исследуемых университетов. Ниже приведена сравнительная гистограмма внешнего окружения университетов, представляющих страны России, США и Великобритании (рис.~\cref{fig:histogramUKRUUS}).

\begin{figure}[ht]
	\centerfloat{
		\includegraphics[scale=0.7]{histogramUKRUUS}
	}
	\caption{Сравнительная гистограмма ресурсов, ссылающихся на университеты России, США и Великобритании.}\label{fig:histogramUKRUUS}
\end{figure}

Анализ сайтов, ссылающихся на сайты ВУЗов, показал, что университеты из одной страны имеют схожее внешнее окружение во многих областях. У всех университетов преобладают такие группы, как Университеты и Научные сообщества. У университетов США и Великобритании эта доля выше, чем у России. Соответственно, в других областях доля российских университетов больше.

Можно сделать вывод о том, что чем выше доля университетов и научных сообществ среди внешних ресурсов университета, тем выше его вебометрический показатель Impact. Этот показатель зависит от количества и качества внешних ресурсов, ссылающихся на сайт. Внешние группы, содержащие университеты и научные сообщества, имеют гораздо больший вес для вебометрического рейтинга, чем другие группы внешних ресурсов.

\paragraph{Общие выводы.} Разработан алгоритм иерархической кластеризации с использованием вероятностного метода понижения размерности многомерных данных Locality-Sensitive Hashing, оптимально подходящий для кластеризации больших коллекций документов (massive datasets). Данный алгоритм был апробирован на тестовой коллекции текстовых документов Reuters. На тестовой коллекции алгоритм показал приемлемую точность (accuracy) и \(F\)-measure в сравнении с классическими методами иерархической кластеризации. Метод кластеризации с использованием LSH значительно превосходит по скорости работы классические методы иерархической кластеризации.

Разработанный алгоритм был использован для анализа веб-пространства нескольких университетов с их окружением.

Получены данные о внешнем окружении веб-пространства университетов России и Великобритании и о ресурсах, ссылающихся на сайты университетов России, США и Великобритании.

\subsection{Теоретико-графовые характеристики в вебометрических исследованиях внутренней топологии крупных сегментов Веба}\label{subsec:ch1/sec4/sub8}

\paragraph{Введение.} В настоящее время стремительно развивается молодое научное направление вебометрика \cite{AlmindIngwersen,Thelwall,Pechnikov}, которая занимается исследованиями количественных и качественных характеристик топологии (гиперссылочной структуры) различных веб-сегментов. Основоположниками таких исследований являются испанская группа Cybermetrics Lab, которая разработала вебометрический рейтинг сайтов различных крупных организаций \cite{RankingWeb} (таких как научно-образовательные учреждения, больницы, научно-исследовательские центры и т. п.).  

Одной из актуальных задач вебометрики является задача анализа внутренней топологии различных сайтов \cite{Thelwall,BlekanovSergeevMaksimov,MaksimovBlekanov,BlekanovSergeevMaksimovBOWTIE}, а также выявление критериев оценки качества веб-сегментов и сравнение этих сегментов по полученным оценкам.  

Для решения вышеуказанной задачи в статье авторами рассмотрены известные теоретико-графовые характеристики веб-графов и разработан комплекс программ на их основе. Данный программный комплекс используется для вычисления указанных выше характеристик, их сравнения на разных веб-сегментах большой размерности, а также для визуализации результатов сравнительного анализа.  

\paragraph{Эксперимент.} В работе ставился эксперимент, в котором разработанный авторами комплекс программ для выявления значимых страниц внутренней топологии крупных веб-сегментов (с помощью используемых теоретико-графовых характеристик) и их визуализации апробировался на крупных сайтах университетского Веба. 

Получение внутренней топологии веб-ресурсов в эксперименте выполнялось с помощью программно-аналитического комплекса для вебометрических исследований, основанного на обобщенном ядре поискового робота \cite{BlekanovSergeevMartynenko} и успешно апробированного в исследованиях \cite{BlekanovSergeevMaksimov,MaksimovBlekanov,BlekanovSergeevMaksimovBOWTIE}.

В качестве теоретико-графовых характеристик сайтов были выбраны несколько мер центральности и связности веб-графов, а именно:
\begin{enumerate}
	\item Степень вершины (Degree Centrality) -- показатель, указывающий для каждой страницы количество страниц, связанных с ней. В веб-графе вычисляются полустепень захода (indegree, количество входящих ссылок) и полустепень исхода (outdegree, количество исходящих ссылок) ссылок \cite{WassermanFaust,OrtegaAguillo}.
		\item Мера центральности Betweenneess -- показатель, указывающий насколько часто данная веб-страница лежит на кратчайшем пути между всеми парами страниц сайта \cite{WassermanFaust,OrtegaAguillo}.
		\item Мера центральности Closeness -- среднее расстояние от данной веб-страницы до всех остальных страниц сайта \cite{WassermanFaust}.
		\item Мера центральности PageRank -- показатель важности страницы. Чем выше ее показатель, тем она важнее \cite{PageBrinMotwani}.
		\item Мера связности p-Cliques -- подграф, который представляет собой полный граф. p -- количество вершин в данном подграфе \cite{OrtegaAguillo}.
		\item Мера связности k-Cores -- подграф, в котором каждая страница связана по крайне мере с k другими страницами в этом подграфе \cite{OrtegaAguillo}.
\end{enumerate}

А также использовались такие общие показатели топологии, как:
\begin{enumerate}
	\item Расстояние (Distance) -- средняя длина всех кратчайших путей в графе \cite{MaksimovBlekanov}. 
	\item Диаметр (Diameter) -- длина самого большого кратчайшего пути в графе \cite{MaksimovBlekanov}.
\end{enumerate}

В качестве сайтов университетского Веба были взяты следующие сайты из Мирового вебометрического рейтинга университетов \cite{RankingWeb}:
\begin{enumerate}
	\item Сайт, занимающий первое место в общем рейтинге:
		\begin{enumerate}
				\item Сайт Гарвардского университета -- ГУ (www.harvard.edu). 
			\end{enumerate}
	\item Сайты, занимающие первые места в рейтинге по Российской Федерации: 
		\begin{enumerate}
				\item Сайт Московского государственного университета -- МГУ (www.msu.ru). 
				\item Сайт Санкт-Петербургского государственного университета -- СПбГУ (www.spbu.ru).
			\end{enumerate}
\end{enumerate}

При получении внутренней топологии выше указанных сайтов учитывались только главные их домены (поддомены не рассматривались). 

Используя вышеуказанный программный комплекс, требовалось получить визуальное представление топологии сайтов университетов в виде композиции наилучших значимых веб-страниц по всем мерам центральности (по каждой мере выбирался топ-10 страниц с наилучшими весами), а также построить и оценить расстояние от главной страницы до этих страниц.

\paragraph{Результаты эксперимента.} В ходе эксперимента для выбранных сайтов университетского Веба были получены следующие значения мер связности и общих показателей топологии (табл.~\cref{tab:uniPagesInnerTopology}):

\begin{table} [htbp]%
	\centering
	\caption{Теоретико-графовые характеристики внутренней топологии университетских сайтов.}%
	\label{tab:uniPagesInnerTopology}% label всегда желательно идти после caption
	\renewcommand{\arraystretch}{1.5}%% Увеличение расстояния между рядами, для улучшения восприятия.
	\begin{SingleSpace}
		\begin{tabulary}{\textwidth}{@{}>{\zz}L >{\zz}C >{\zz}C >{\zz}C@{}} %Вертикальные полосы не используются принципиально, как и лишние горизонтальные (допускается по ГОСТ 2.105 пункт 4.4.5) % @{} позволяет прижиматься к краям
			\toprule     %%% верхняя линейка
			Теоретико-графовые характеристики & ГУ & МГУ & СПбГУ\\
			\midrule %%% тонкий разделитель. Отделяет названия столбцов. Обязателен по ГОСТ 2.105 пункт 4.4.5
			Количество страниц & 451 & 45 557 & 51 945 \\				
			Количество ссылок, принадлежащих главному домену & 27 615 & 2 043 221 & 4 862 750 \\
			Общее количество ссылок & 72 355 & 2 143 474 & 5 017 815 \\
			Расстояние & 3.04 & 7.19 & 6.02 \\
			Расстояние & 7 & 30 & 18 \\
			Среднее значение indegree, outdegree & 13.59 & 28.1 & 69.79 \\
			Количество вершин в p-Cliques & 26 & 10 & 144 \\
			\bottomrule %%% нижняя линейка
		\end{tabulary}%
	\end{SingleSpace}
\end{table}

Композиция всех наилучших (по заданным мерам центральности) значимых веб-страниц исследуемых сайтов представлена на рис.~\cref{fig:harvardUComposition}-\cref{fig:spbUComposition}. На рис.~\cref{fig:harvardUComposition}-\cref{fig:spbUComposition} введены следующие обозначения цветов: 
\begin{itemize}
	\item Красный -- главная страница сайта; 
	\item  Зеленый -- полустепень исхода; 
	\item Бирюзовый -- полустепень захода; 
	\item Синий -- мера Betweenness; 
	\item Розовый -- мера Closeness; 
	\item Оранжевый -- мера PageRank.
\end{itemize}

\begin{figure}[ht]
	\centerfloat{
		\includegraphics[scale=0.48]{harvardUComposition}
	}
	\caption{Композиция всех наилучших значимых веб-страниц сайта ГУ.}\label{fig:harvardUComposition}
\end{figure}

\begin{figure}[ht]
	\centerfloat{
		\includegraphics[scale=0.5]{moscowSUComposition}
	}
	\caption{Композиция всех наилучших значимых веб-страниц сайта МГУ.}\label{fig:moscowSUComposition}
\end{figure}

\begin{figure}[ht]
	\centerfloat{
		\includegraphics[scale=0.46]{spbUComposition}
	}
	\caption{Композиция всех наилучших значимых веб-страниц сайта СПбГУ.}\label{fig:spbUComposition}
\end{figure}

\paragraph{Общие выводы.} По результатам эксперимента можно сделать следующие выводы:
\begin{enumerate}
	\item Сайт ГУ представляет собой навигационный сайт, перенеся большую часть информации на свои поддомены (ссылки главного домена составляют \textit{38.2\%} от общего числа ссылок). Данное явление подтверждается тем, что веб-страницы по полустепени исхода у ГУ лежат близко к главной странице (глубина 2-4), в то время как у МГУ и СПбГУ они лежат на определенном удалении (8-11 у МГУ и 6-9 у СПбГУ).
	\item Показатели СПбГУ (такие как расстояние, диаметр, количество вершин в p-Cliques, положение значимых вершин относительно начальной страницы) лучше, чем таковые у МГУ. Это значит, что сайт СПбГУ показывает более высокую связность, нежели сайт МГУ.
	\item Авторитетные страницы расположены на следующем уровне глубины после главной страницы, о чем говорит их высокий вес PageRank.
	\item Значимые веб-страницы по мере Betweenness у всех сайтов лежат в радиусе пяти ссылок от главной страницы, однако у сайта МГУ они расположены последовательно. Это объясняется тем, что множество кратчайших путей проходит через эти страницы. При отказе одной из таких страниц увеличиваются размеры множества кратчайших путей, а также появляется риск полной потери связей с множеством страниц.
	\item Значимые веб-страницы по мере Closeness у ГУ лежат на уровнях 1 и 3, у МГУ -- на уровне 4 и 5, у СПбГУ -- на уровне 3. Это показывает, что рассмотренные сайты централизованы недалеко от главной страницы.
\end{enumerate}

Также эксперимент показал некоторую особенность главных страниц рассмотренных сайтов, а именно:
\begin{itemize}
	\item главная страница ГУ не попала ни в один из топ-10;
	\item главная страница МГУ попала лишь в топ-10 по степени захода;
	\item главная страница СПбГУ попала сразу в три топ-10: по степени захода, по мере Betweenness и по мере PageRank. 
\end{itemize}

В дальнейшем планируется расширить эксперимент, проверив эргономические параметры значимых веб-страниц, выделенных программным комплексом, у заданных сайтов. 

\FloatBarrier
