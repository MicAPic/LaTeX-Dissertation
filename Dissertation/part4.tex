\chapter{Конструирование виртуальных мультиагентных подсетей}\label{ch:ch4}

\section{Конструирование Web-ресурсов}\label{sec:ch4/sect1}

\section{Типы виртуальных сетевых агентов}\label{sec:ch4/sect2}

\subsection{Index data structure, functionality and microservices in thematic virtual museums}\label{subsec:ch4/sec2/sub1}

\paragraph{1. Introduction.} Museum as nonprofit institution which responsible for preserving historical and cultural heritage, exhibit tangible and intangible to public is facing huge problem in digital era as mention in \cite{AnggaiBlekanovSergeev2014}. Museums institution are starting digitizing their collections and describe detail information in order to intensify their activities and present on the Internet as a part of public services. However, digital transformation is inevitable, museums institution must provide good impact, experiences and values to their visitors.

The thematic concept in virtual museum is one of the approaches we take in order to accelerate digital transformation of museums institution. We are combining virtual museum (VM) and information retrieval (IR) concepts in conducting information management, analytics, and focus on users-centric to enhance their experience by providing visitors’ information need within the virtual museums system.

In this research work we are preparing, investigating and conducting experiments on data access service, index construction, applying ranking algorithms and development of microservices in thematic virtual museums for speeding up disruptive innovation in museums institution.

\paragraph{2. Data crawling.}  In section 2 we will describe how this application data access crawling service get the content of web pages. This crawler service can take data from museums institution in text, Hypertext Markup Language (HTML), JavaScript Object Notation (JSON), or Extensible Markup Language (XML) form. We define specific web Uniform Resource Locator (URL) to crawl and run the service to obtain content pages. The crawler service maintains and run a job for specific website and not for general uses, because it ought to crawl based on characteristics of each page in those websites. As in our case we are getting the data from Registration Museum under Ministry of Education and Culture, Republic of Indonesia. There are two models web pages structure in order to get full museum collection pages as the following.

\begin{enumerate}
	\item List of collections. The crawler service directly accesses to the URL which contains list of museum collections. This URL address show 20 items per page and navigation page to identify how many pages still remain as previous or next page from the first to the end page. All URL retrieved from this page which identified as detail collection URL will be adding to download-queue.
	\item Detail of collection. This page contains detail collection information and it can be marked as end of page to be crawled. The information which contain on this page is name of museum institution, collection type, function, age, description and picture with a title.
\end{enumerate}

The crawler services automatically download the content page until all pages have been crawled. In order to prevent Internet network traffic, we are providing delay-time parameter to the crawler services, therefore this engine can be customizing to aggressive or polite crawler modes. The algorithm we are using on this task is Breadth First Search \cite{HassaanBurtscherPingali}, however the page of collection is not deep, therefore to download those pages the formula can be written as shown below:
\[
f(p) = \sum_{i = 1}^{n} \sum_{j = 1}^{m} \textit{dl }(\textit{page})
\]

\paragraph{3. Parsing data.} In this stage the service will parse data which has been obtained. There are two models we have been using in order to parsing data from crawler service, first we are directly parsing HTML format into Thermal VM (TVM) data schema, and second the parser service follow standard rule for data exchange. Parser service has developed for performing tasks in general, therefore data collections which obtained by crawler can be fully implemented on this service. We are using method as shown in Figure~\cref{fig:tvmMapping}.

\begin{figure}[ht]
	\centerfloat{
		\includegraphics[scale=1]{tvmMapping}
	}
	\caption{Extracting and mapping dynamic schema to TVM schema.}\label{fig:tvmMapping}
\end{figure}

The parser service in Figure~\cref{fig:tvmMapping}, show step by step to exchange data where this service can be receiving and parsing XML or JSON format in dynamic schema includes Conceptual Interchange Documentation -- Conceptual Reference Model, Lightweight Information Describing Objects (LIDO), and Open Archives Initiative Protocol for Metadata Harvesting (OAI-PMH), however, we must define dynamic or unknown schema and extracting this schema to well form. Data exchange can be starting when the well form schema is similar with TVM schema, if not then mapping the well form schema to TVM schema should be done, only then data exchange can be performed through Hypertext Transfer Protocol (HTTP) web services.

\paragraph{4. Data store.} An important part of data-access service is storing and maintaining data collections which have been crawled and parsed to the standard data structure form. There are many techniques to preserve document collections, one of them is storing into databases. To store data from crawler service, we are using non-traditional database key- value MongoDB for handling structure, semi-structure and unstructured data. There are many advantages of MongoDB as mention in \cite{StanescuBrezovanBurdescu,JiaZhaoWang,WuChenJiang}, this type NoSQL document-oriented database using JSON-like format called Binary JSON (BSON), support for partition and MapReduce. MongoDB is using document store model, allow developer to create free-schema, running on multiplatform and opensource.

In our case we have developed a system to access MongoDB database server using Mgo (mango) driver for Go programming language. The Mgo driver is providing simple application programming interface, easy to use, and fast enough for performing task such as create, read, update, delete operations (\textit{https://labix.org/mgo}). The design of TVM database schema for museum collections \cite{AnggaiBlekanovSergeev2015} as shown in table~\cref{tab:tvmDbSchema} is independent and can be customized depend on the goals of spesific index task.

\begin{table} [htbp]%
	\centering
	\caption{TVM database schema for document collections.}%
	\label{tab:tvmDbSchema}% label всегда желательно идти после caption
	\renewcommand{\arraystretch}{1.6}%% Увеличение расстояния между рядами, для улучшения восприятия.
		\begin{adjustbox}{width=0.8\textwidth}
		\small
			\begin{tabulary}{\textwidth}{@{}>{\zz}L >{\zz}C >{\zz}C@{}}% Вертикальные полосы не используются принципиально, как и лишние горизонтальные (допускается по ГОСТ 2.105 пункт 4.4.5) % @{} позволяет прижиматься к краям
			\toprule     %%% верхняя линейка
			N & Key & NameDescription \\
			\midrule %%% тонкий разделитель. Отделяет названия столбцов. Обязателен по ГОСТ 2.105 пункт 4.4.5
			1 &  \_id &  Unique object identification \\ 
			2 & iddata & Identification of each data sources \\
			3 &  idinstitution &  Identification of museum institution\\ 
			4 & name & Object name or title \\ 
			5 & regcode & Registration code\\
			6 & category & Information about category or type of the object \\
			7 & collector & Person who collects the object \\ 
			8 & datefound & Date when the object was founded \\
			9 & placesfound & Places where the object was founded \\
			10 & period & Period or year of the object event happened \\
			11 & age & Age of the object \\
			12 & dimensions & Dimensions of the object\\
			13 & weight & Weight of the object \\
			14 & material & Materials forming or made of the object \\ 
			15 & condition & Description of past and recent object condition \\
			16 & totalcollection & Total object in the museum institution \\
			17 & description & Detail object description \\
			18 & ref & Reference information about where the data have been taken\\
			19 & creator & Creator of the document record in database\\
			\bottomrule %%% нижняя линейка
		\end{tabulary}%
	\end{adjustbox}
\end{table}

\paragraph{5. Forward index.} There are several IR techniques for document indexing system (\textit{https://www.elastic.co/blog/found-indexing-for-beginners-part3}), one of them is forward index \cite{BrinPage}, which is fast to perform task in document indexing. We are taking the museum collections data from MongoDB and managing those collections in TVM forward index. We have modified standard forward index data structure according to TVM data structure needed as shown in Figure~\cref{fig:tvmForwardIndex}.

\begin{figure}[ht]
	\centerfloat{
		\includegraphics[scale=1]{tvmForwardIndex}
	}
	\caption{Extracting and mapping dynamic schema to TVM schema.}\label{fig:tvmForwardIndex}
\end{figure}

TVM forward index is pair document-payload using unique document identifier (DocId) for each document collections, in order to gain fast access to a document we are using hash map. Payload or embedding information in this forward index can be contains document identifier, list of terms contains in document, a record document information from database, and some necessary information. The development on flexible payload as our approach in TVM forward index give us space to embed something useful in managing and interacting with TVM inverted index.

\paragraph{6. Inverted index.} TVM concept which focus on fastest retrieving information need must be designed with good data structure, therefore we have created inverted index as a base for maintaining and searching information from any type application or users queries. There are many standard inverted index models have studied in \cite{ZobelMoffat,ManningRaghavanSchutze,PanevBerberich}, based on those state of the art we are modifying and designing inverted index structure for TVM need \cite{AnggaiBlekanovSergeev2017} as shown in Figure~\cref{fig:tvmInvertedIndex}.

\begin{figure}[ht]
	\centerfloat{
		\includegraphics[scale=1]{tvmInvertedIndex}
	}
	\caption{Extracting and mapping dynamic schema to TVM schema.}\label{fig:tvmInvertedIndex}
\end{figure}

TVM inverted index maintain terms in dictionary using hash map in order to gain fast lookups, adds and deletes operations. Hash map as theoretically is performing \(O(1)\) and the worst case is \(O(n)\). Posting list for each term in dictionary also using hash map, it means that our dictionary term design is using double hash map. Access to an embedded information in this TVM inverted index will perform \(O(1)\) and the worst case is \(O(1*n)\) or \(O(n + m)\) respectively.

TVM inverted index module is providing several ranking functions such as bag of words term-frequency -- inverse document frequency (TF-IDF) \cite{ManningRaghavanSchutze}, cosine similarity measure vector space model (VSM) \cite{SaltonBuckley}, includes data provider for conceptual modeling latent semantic indexing (LSI) [9, 13] and topic modeling latent derelict allocation (LDA) \cite{BleiNgJordan}, another ranking function can be added as needed as long as it is still matching with TVM inverted index data structure design.

\paragraph{7. Microservices.} Index service in TVM are running on server side, therefore we have designed a microservices in order to communicate between applications, and serve requests from front-end to back-end server \cite{AnggaiBlekanovSergeev2015}. There are many benefits of microservice architecture as mentions in \cite{Singleton,GuoWangZeng,Bakshi}. TVM microservice in this case only support HTTP POST request with data submission to prevent security issue as shown in algorithm below:

\begin{figure}[ht]
	\centerfloat{
		\includegraphics[scale=1]{algoQuery}
	}
	\label{fig:algoQuery}
\end{figure}

Algorithm pseudocode is describing how TVM microservice listening requests from clients, processing the request and give back a response in appropriate JSON format. HTTP header authentication schema is Basic, which transmit user and password credentials when the connection has established. The client request for searching similarity document is identified by DocId where each request will trigger inverted index ranking function and return list-docIds no more than maximum top-K ranked. The list-docId will be adding to the index-cache and using it again on the similar DocId request for reducing similar calculation process.

\paragraph{8. Experiment and results.} TVM system which have provided a design for microservices in this experiment will be used for handling many requests from front-end to back-end server. We have used Intel Xeon Processor 2620v4, memory DDR4 32 Gb, and hard disk Skyhawk Surveillance 2 terabyte in conducting this experiment. The data we have been using as dataset are from “Collection Museum Registration System”, Directorate Cultural Heritage Preservation and Museum, Ministry of Education and Culture, The Republic of Indonesia, which contain 29 362 collections.

In this experiment we have prepared three methods which all outputs will be encoded, written and return to the client side in JSON format, first the service will return only list of docIds form to the client in consequently allowing them to process those list of docIds for another tasks, second the service will get the list of docIds and using them as keys to identify full detail collections information in TVM forward index payload, third the service will get the list of docIds and using them to perform a query to database mongoDB for searching the detail collections information. The query request method to indicate service performance is using document at a time (DAAT) as shown in Figure~\cref{fig:daatQuery}.

\begin{figure}[ht]
	\centerfloat{
		\includegraphics[scale=1]{daatQuery}
	}
	\caption{DAAT query requests from front-end to back-end server.}\label{fig:daatQuery}
\end{figure}


DAAT query request from client or front-end using appropriate JSON structure in Figure 4 was handled by TVM independent microservices where after query has performed, then inverted index function directly calculates documents score and the function returned list of top-K ranking in list of docIds form. The query processing time for accessing information to database is 84.5\% and access to index payload only need 0.02\% from total query request time.

The second experiment, we are using query term at a time (TAAT), where each term in TVM inverted index dictionary have been using for query requests in order to evaluate and calculating documents score which have contained in postings list. The result as shown in Figure~\cref{fig:taatQuery} is describing query time to database need 99.4\% and query to TVM index payload only 0.0005\% from total query request time.

\begin{figure}[ht]
	\centerfloat{
		\includegraphics[scale=1]{taatQuery}
	}
	\caption{TAAT query requests from front-end to back-end server.}\label{fig:taatQuery}
\end{figure}

The first method is only gives docIds list rather than give a full information, therefore there is no request needed to database or index payload, beside this docIds list can be used as caching for top-K docIds ranking in order to reduce computation time in back-end server. The cache key is docId of recent document which is using as query request, and the value is containing array set of documents identification. The second and third methods give us a significant result, where our method have provided an access to collaborate between TVM inverted index and forward index by given embedded information as payload. Our experiments shown that the methods have been proposed can reduced time to access a detail information of the collection rather than given many direct queries to database system.

\paragraph{9. Conclusion}. In this work we have constructed data access service, modified forward and inverted index, and design special microservice architecture for TVM to provide relevant information for the virtual museums visitors. There are several experiments we have conducted, and shown our methodology give significant results when exchanged an information and collaborated between TVM forward and inverted index, query request, process and response through microservices give high performance output. Dynamic or flexible payload in TVM indices structure can be used for multipurpose indexing method in the development of modern information retrieval.

\subsection{Construction inverted index for dynamic collections visualization in thematic virtual museums system}\label{subsec:ch4/sec2/sub2}

\subsubsection{Introduction}

hematic virtual museum system main goal is providing more relevant information based on user characteristic, behavior and desire. Thematic exhibition as main unique feature is very considering on users’ whole activities. Thematic concept can increase user experience, deep knowledge, discovery new related content, explore unexplored, and most important thing is not disturb users by providing irrelevant information.

Barry and Maria define the creation of new knowledge, transformative and self-directed experiences, engagement with the full diversity of visitors, and transparency as the source of a viewpoint of the exhibition as museum specific evaluation criteria. Museum exhibition is not a book on the wall rather than leading visitors to new attitudes, values and ideas even in very different cultural backgrounds or religious beliefs \cite{LordPiacente}.

Interpretation of tangible and intangible cultural and natural heritage can be spread in various recent devices of information and communication technology to the end-users. The system initiative to serve users by providing accurate museum collections information which can be delivered through desktop, web or mobile application in real time. The system can help curator or museum administrator for research, clarify, grouping, visualizing and design thematic event exhibition in some cases.

John H Falk writes that most museum visitors mentioned that visitor go to museums in their free time in order to have fun and enjoy themselves or to see new interesting things in a relaxing and aesthetically pleasing setting \cite{Falk}.

The system with interactive information in the thematic exhibition which has been processing and delivering to the user will be leading to the better understanding by traveling in similar fields i.e. art, archeology, artefact and culture value. Bridging entities relationship by weighting and combining it in vector space also a part of this research.

In order to accomplish those tasks, this research has been performed and constructed thematic virtual museums inverted index to gain fastest retrieval document collections.

\subsection{Indexing System}

Inverted index usually known as inverted file is a data structure to store document-term pair which widely used in information retrieval field. Christopher D. Manning define Information retrieval (IR) in academic field study as a technique to find material (usually documents) of an unstructured nature (usually text) that satisfies an information need from within large collections (usually stored on computers) \cite{ManningRaghavanSchutze}. Inverted index often used to process many data form, i.e. structure, semi- structure and unstructured.

Previously front-end interface has been created and it can manage, grant and revoke data access from and to the system using data access layer. There are 105 museums with total 24.608 collections from Ministry of Education and Culture, The Republic of Indonesia. Each collection in museum institutions have structure and unstructured data source. This data sources increase day by day depend on museum administrator/curator which managing every museum institution. Those data become hard to maintain just by using the recent technology of database system, therefore it is necessary to develop a system for extracting and processing data sources until become important information and useful for visitors \cite{AnggaiBlekanovSergeev2015}.

The actual concern for managing big collections in museum institutions are gaining fast searching relevant information document collections in thematic virtual museums system, therefore this work has constructed an inverted index which aligned with thematic virtual museums information need \cite{AnggaiBlekanovSergeev2014}.

\paragraph{A. Forward Index}
Forward index task usually using document-term, the structure is very helpful for retrieving a set of information as a payload in order to get fast performance related to virtual museum item collections. The context or information as payload in this index automatically will be processing the documents after data access layer crawler service has finished the job schedule.

\paragraph{B. Inverted Index}
An inverted index is a data structure that maps a word, or atomic search item, to the set of documents, or set of indexed units, that contain that word -- its postings. Operational IR also concerns on block update speed, access speed, index size, dynamics and scalability as mentioned in \cite{CuttingPedersen}. The index data structure contains a pair of term-document or inverted list for each term stores a list of the documents \(d\) \cite{MoffatZobel}.

\subsubsection{Thematic Virtual Museum Index}

The system is using special architecture and multi system indexing in virtual museum inverted index. This index can manage collections in many forms of data structured, semi structured or unstructured which have processed by a data access layer engine.

There are several tasks to perform inverted index in this virtual museum system such as tokenizing, removing stop word, stemming and process to index structure.

Museum index assigns unique Id for each document in the collections as document identifier (docId). Construction of dictionary term makes this system fast to search every cooccurrence document in the specific term as a key. In order to gain the best performance this index used the Hash table to store terms to perform \(O(1)\) lookup time according to \cite{CuttingPedersen}.

\paragraph{A. Embedding, Loading, and Storing}
Embedding some information as payload in inverted index data structure is one of the main concern related to the specific area in virtual museum, this information will be stored together with inverted-list. There are several techniques for compressing inverted lists \cite{ManningRaghavanSchutze,MoffatZobel}. The thematic virtual museum inverted index store data in file format which contain key-term and sorted \(\textit{docId}\) by using compressing technology Varint encoding \cite{LemireBoytsov} for reducing file size on the disk. This system using Solid-State Drive (SSD) for fast read/write file compared to Hard Disk.

\paragraph{B. TF-IDF Weighting}
Thematic virtual museum inverted index using Term Frequency (TF) and Inverse Document Frequency (IDF) \cite{ManningRaghavanSchutze} as base scoring method to determine important information in museum collections. The output of this statistical calculation will be sorted as ranking from highest to lowest.

\paragraph{C. Vector Space Model (VSM)}
Similarity measure is better than traditional TF-IDF, the cosine measure which each document is assigned a numeric score indicating similarity with the query, and then the documents that score the highest are displayed as answers \cite{MoffatZobel}. Salton and McGill define cosine vector similarity formula between query and document as shown \cite{SaltonBuckley}.
\[
\textit{cosine}(q, d) = \frac{\sum_{k=1}^{t} w_{qk} \cdot w_{dk}}{\sqrt{\sum_{k=1}^{t} w_{qk} \cdot \sum_{k=1}^{t} w_{dk}}}
\]

The thematic virtual museum inverted index library implements this cosine similarity method as a part of modules as a complement for the fast searching document by comparing the similarity of users’ query.

\paragraph{D. Index Adaptive Module}
The system engine design which has been developed contain several independent program modules to perform the tasks, this index can be integrated or serve another complex adaptive ranking retrieval system without modifying virtual museum inverted index core.

Modification of standard inverted index structure and adding some functions allowing this system to process information need more complex by combining fastest term-document pair for building another retrieval structure.

\subsubsection{Experiment and Results}
In order for testing this virtual museum inverted index, the engine has crawled museum information from “Collection Museum Registration System”, Directorate Cultural Heritage Preservation and Museum, Ministry of Education and Culture, The Republic of Indonesia, which contain 24.608 items.

\begin{table} [htbp]%
	\centering
	\caption{TVM database schema for document collections.}%
	\label{tab:documentCollections}% label всегда желательно идти после caption
	\renewcommand{\arraystretch}{1.6}%% Увеличение расстояния между рядами, для улучшения восприятия.
		\begin{tabulary}{\textwidth}{@{}>{\zz}L >{\zz}C >{\zz}C@{}}% Вертикальные полосы не используются принципиально, как и лишние горизонтальные (допускается по ГОСТ 2.105 пункт 4.4.5) % @{} позволяет прижиматься к краям
			\toprule     %%% верхняя линейка
			No & Type & Item \\
			\midrule %%% тонкий разделитель. Отделяет названия столбцов. Обязателен по ГОСТ 2.105 пункт 4.4.5
			1 &  Numismatic / Heraldic &  2746 \\ 
			2 & Ethnography & 10107 \\
			3 & Ceramics &  4108 \\ 
			4 & Fine Art / Handicraft & 1516 \\ 
			5 & Philology & 2902 \\
			6 & History & 1394 \\
			7 & Archeology & 1153\\ 
			8 & Technology / Modern  & 682 \\
			 \multicolumn{2}{c}{\makecell{Total}} & 24608 \\
			\bottomrule %%% нижняя линейка
		\end{tabulary}%
\end{table}

Those document collections which have been obtained then stored in MongoDB database. The next step, the information is processed and assign unique \(\textit{docId}\) documents into inverted index, the result as shown in Fig.~\cref{fig:timeProcessingInvertedIndex}.

\begin{figure}[ht]
	\centerfloat{
		\includegraphics[scale=1]{timeProcessingInvertedIndex}
	}
	\caption{Time processing document collections into inverted index.}\label{fig:timeProcessingInvertedIndex}
\end{figure}

Processing documents and input into virtual museum inverted index for 24.608 collections require processing time approximately 2.29 seconds by using Intel Xeon Processor 2620v4, memory DDR4 32Gb, and Kingston Savage SSD 256Gb. This index intends to be used for serving all user’s activities in front end server. Further, this inverted index has tested by using query document to document as this research main goal to approximate thematic collections related to user’s activity and document collections.

\begin{figure}[ht]
	\centerfloat{
		\includegraphics[scale=1]{timeProcessingDocumentQuery}
	}
	\caption{Elapsed time processing document to document query.}\label{fig:timeProcessingDocumentQuery}
\end{figure}

The result in Fig.~\cref{fig:timeProcessingDocumentQuery}, show that given a query “a document to a document” require time less than one microsecond per query, this is aligned with the main goal to design virtual museum inverted index data structure to serve more fast and relevant.

\subsubsection{Visulization}
\paragraph{A. Collections in Map}
Information visualization is one of concern in thematic exhibition in order to deliver and serve high quality experience to the visitors.
Design map plan and leveling will help visitors to classify information which has previously received from information traveling visualization. Let visitor move to the area inside the map which could be formed to the internal or external exhibition with there are no predefine items have assigned. Providing more space as a proportion to give object transformation is considerable as part of esthetics. The room design maybe contains object on the floor, wall or in some neared-location or can be reached by users as shown in Fig.~\cref{fig:designExplorationRoom}.

\begin{figure}[ht]
	\centerfloat{
		\includegraphics[scale=1]{designExplorationRoom}
	}
	\caption{Design Exploration Room for Collection Exhibition.}\label{fig:designExplorationRoom}
\end{figure}

Give the floor planning with good navigation will make users easier to use this system. In virtual it can be providing area by grouping the object, even this particular area didn’t exist on the real museum institution.

\paragraph{B. Dynamic Collection Visualization}
When the users visiting a virtual museum, the system should determine what is the information trying to convey? what the motivated topics which make them enjoy and satisfied? In this situation, thematic virtual museum trying to create the system to maintain interactive information. The collections will appear after automatic user query have been processed. The exhibition which has been displayed should be drawn users’ attention and make more attractive by triggering with relevant collections visualization.

The degree of similarity will be leading user from room to room or an area to area of exhibition as shown in Fig.~\cref{fig:dynamicCollectionsArea}. It can be pre-labeling data as a temporary or a permanent exhibition for each object in spread area. The system will be responsible for redisplaying contents by marking user traversed area.

\begin{figure}[ht]
	\centerfloat{
		\includegraphics[scale=1]{dynamicCollectionsArea}
	}
	\caption{Dynamic Collections Area.}\label{fig:dynamicCollectionsArea}
\end{figure}

Creating specific functional area design for scheduled thematic event exhibition is recommended to perform flexible objects which have determined include newly arrived collection in some museums. Along the process, user respond and time spent in the room or area where collections appeared must be recorded as a feedback parameter. User also can be projecting some group of collections base on a manual query to the system with or without their base information.

The dynamic collection will collect visitor visit experiences in order to be processed near real time or next time when visitors will come back again on this system. Trying to understand the visitors to far and more complex by defining some variables is very useful in background inverted index process. Quantitative measures such as demographics provide too little information about visitors in relation to museums to be a useful variable for describing and understanding the museum visitor experience \cite{Falk}.

\paragraph{C. User Interaction}
User interaction always recorded by the system on the fly and this information in near real-time will be automatically processed in the background. The result of those process will generate more accurate information related to visitor characteristics, behavior and desire. Defining embedded information into inverted index is helpful for delivering more accurate exhibitions in the specific fields.

When users are walking around the space, the system provides a feature to control some object which has an ability to be controlled, this approach is mean to increase user sensitivity and reaction to gain new knowledge or discovery. Design direct or indirect appearance of the objects can be projected over a period of time. Understanding user needs and wants is critical for exhibition space to be not only attractive and effective for display but also accessible \cite{LordPiacente}.

Working to design ideal interface to the end-user is very crucial things, affected visitor will return another time if visitors satisfied using the application or will be leaving because some of poor user interface design, those all approximation for design user interface visualization.

\subsubsection{Conclusion and Future Work}
In this paper showed a design and modification of the inverted index experimental result for thematic virtual museums system and a sample of exhibition visualization to get more precision by embedding method. This research will continue to implement this index from contextual to concept by using latent semantic indexing (LSI) projection in future work and improving user interface for more attractive.

\subsection{Design Muntoi web-based framework and search engine analytics for thematic virtual museums}\label{subsec:ch4/sec2/sub3}

\subsection{Management and information processing at virtual museum}\label{subsec:ch4/sec2/sub4}

\subsection{Modification biterm topic model input feature for detecting topic in thematic virtual museums}\label{subsec:ch4/sec2/sub5}

\section{Оптимальное конструирование сборщика (цепочка роботов)}\label{sec:ch4/sect3}

\section{Реализация программного комплекса}\label{sec:ch4/sect4}

\FloatBarrier

